\chapter[Considerações Finais]{Considerações Finais}
A gamificação possui várias técnicas que propiciam que uma dada aplicação
consiga gerar motivação e engajamento dos usuários a partir de vários sentimentos
básicos, que são divididos nas motivações básicas. Isto possibilita que os
usuários permaneçam motivados e engajados.

Percebi que o processo de gamificação é importante para proporcionar
envolvimento dos usuários em pontos que a aplicação não possui por padrão,
fazendo com que este esteja motivado ao longo das fases do
ciclo de vida do software.

Ao longo do período proposto, o trabalho foi desenvolvido e todas as motivações
básicas propostas, com exceção de Escassez e Curiosidade, foram implementadas na Rede Social About. 
Estas duas motivações não foram implementadas pois a essência da rede social
já motiva os usuários nestes pontos. Foi possível aplicar todas essas motivações básicas
através de implementações de algumas técnicas, através com o \textit{Framework} \textit{Octalysis}.

O processo de medição dos resultados da aplicação da gamificação na RSA não foi executado,
assim como o acordado na apresentação da primeira etapa deste trabalho. Acordamos
que o escopo estava muito grande e não haveria tempo hábil para a defesa até meados de
Dezembro de 2017.

O piloto foi lançado, os dados foram recolhidos, as análises estatísticas foram feitas,
foi desenhado o novo \textit{Framework} de gamificação, possibilitando assim que a RSA
fosse gamificada.
Dessa forma, o trabalho foi concluído com sucesso, aplicando na RSA todas as motivações básicas, 
implementando um conjunto de técnicas de cada motivação na plataforma.
Foi possível observar que dentro do ciclo de vida de uma aplicação, é necessário que o usuário
esteja engajado e motivado em todas as suas fases. E a RSA tinha uma falha neste ponto,
visto que as fases de dia a dia e fim de jogo não apresentavam muitos aspectos motivacionais
do ponto de vista do seu criador.
Logo, uma das propostas de solução é a aplicação da gamificação, que foi utilizada neste
desenvolvimento tecnológico.

%
% Com a elaboração deste trabalho, consegui observar
% a eficácia do plano pedagógico aplicado ao curso de Engenharia de \textit{Software}.
% Verifiquei que todo o ciclo de \textit{Software}
% sendo estudado e mantido ao longo da graduação, aplicando seus conceitos
% no planejamento, elicitação de requisitos, metodologias de desenvolvimento,
% implementação, verificação e validação, entre outras áreas.
%
% O desenvolvimento de uma rede social transpassa todas as etapas do
% ciclo de desenvolvimento de \textit{Software}, fazendo com que os conhecimentos
% adquiridos ao longo da graduação fossem utilizados.
%
% Foi possível observar a importância da gamificação para a conquista
% de motivação e engajamento por parte dos usuários com alguns objetivos
% definidos. Assim, é possível observar a eficácia da aplicação do 
% \textit{Framework} \textit{Octalysis} dentro da \textit{Rede} \textit{Social} \textit{About}.
