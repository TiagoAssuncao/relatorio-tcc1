\chapter[Considerações Finais]{Considerações Finais}

Com a elaboração deste trabalho, foi possível válidar
a eficácia do plano pedagógico aplicado ao curso de Engenharia de \textit{Software}.
É possível observar a eficácia de todo o ciclo de \textit{Software}
sendo estudado e mantido ao longo da graduação, aplicando seus conceitos
no planejamento, elicitação de requisitos, metodologias de desenvolvimento,
implementação, verificação e validação, entre outras áreas.

O desenvolvimento de uma rede social transpassa todas as etapas do
ciclo de desenvolvimento de \textit{Software}, fazendo com que os conhecimentos
adquiridos ao longo da graduação fossem utilizados.

Foi possível observar a importância da gamificação para a conquista
de motivação e engajamento por parte dos usuários com alguns objetivos
definidos. Assim, é possível observar a eficácia da aplicação do \textit{Framework}
\textit{Octalysis} dentro da \textit{Rede} \textit{Social} \textit{About}.
