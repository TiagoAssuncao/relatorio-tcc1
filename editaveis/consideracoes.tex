\chapter[Considerações Finais]{Considerações Finais}
A gamificação possui várias técnicas que propiciam que uma dada aplicação
consiga gerar motivação e engajamento dos usuários a partir de vários sentimentos
básicos, que são divididos nas motivações básicas. Isto possibilita que os
usuários permaneçam motivados em fases da aplicação que naturalmente não os
engajam.

Assim, percebi que o processo de gamificação é importante para propiciar
envolvimento dos usuários em pontos que a aplicação não possui por padrão,
fazendo com que este esteja motivado ao longo das fases do
ciclo de vida do software.


Com a elaboração deste trabalho, consegui observar
a eficácia do plano pedagógico aplicado ao curso de Engenharia de \textit{Software}.
Verifiquei que todo o ciclo de \textit{Software}
sendo estudado e mantido ao longo da graduação, aplicando seus conceitos
no planejamento, elicitação de requisitos, metodologias de desenvolvimento,
implementação, verificação e validação, entre outras áreas.

O desenvolvimento de uma rede social transpassa todas as etapas do
ciclo de desenvolvimento de \textit{Software}, fazendo com que os conhecimentos
adquiridos ao longo da graduação fossem utilizados.

Foi possível observar a importância da gamificação para a conquista
de motivação e engajamento por parte dos usuários com alguns objetivos
definidos. Assim, é possível observar a eficácia da aplicação do \textit{Framework}
\textit{Octalysis} dentro da \textit{Rede} \textit{Social} \textit{About}.
