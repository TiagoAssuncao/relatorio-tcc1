\chapter[Metodologia]{Metodologia}

Para a definição do trabalho, os seguintes
passos serão estabelecidos a fim de executar proposta, cada um será
descrito em uma subsessão: execução do teste piloto , survey para identificação das técnicas de gamificação,
Análise estatística das técnicas, construção do framework de gamificação, escolha do objeto de gamificação,
implementação das técnicas no objeto de gamificação.

\section{Execução do Piloto}
\label{sub:execu_o_do_piloto}
A ideia sobre a rede social surgiu de uma simples ideia de um aluno de graduação. Desta forma, se até mesmo a
ideia era impalpável, algumas perguntas surgem sobre a gamificação desta, por exemplo:

\begin{itemize}
    \item Como conseguir averiguar como traços de gamificação para que esta seja aplicada?
    \item Como identificar quais são as motivações básicas que devem ser levadas em consideração?
    \item Para atingir tais motivações, quais são as técnicas que devem ser adotadas?
\end{itemize}

Para responder essas perguntas, que irão servir de insumo para o direcionamento dos pilares da gamificação, será elaborado um projeto piloto, tal qual terá algumas funcionalidades básicas da RSA, sem levar em consideração requisitos não funcionais como: segurança, usabilidade, performance e padrões de design.

O objetivo é proporcionar aos usuários um cenário similar, quanto a funcionalidade, ao que estes irão utilizar
na RSA, a fim de identificar quais as técnicas de gamificação são mais presentes.

Para desenvolver e aplicar este piloto, será necessário executar um procedimento com os seguintes passos:
% definir tecnologia,
% desenvolver a solução, virá-la para produção, aplicar marketing da solução para um público reduzido, manter solução, finalizar solução.

\begin{itemize}
    \item Definir tecnologia
    \item Desenvolver solução
    \item Virá-la para produção
    \item Aplicar marketing da solução para um público reduzido
    \item Manter solução
    \item Finalizar solução
\end{itemize}

Estes pontos necessários para a implementação da solução do projeto piloto serão detalhados 
nas subsessões a seguir:

\subsection{Definir Tecnologia}
\label{sub:definir_tecnologia}
Como o desenvolvimento da solução para o projeto piloto não necessita conter interfaces com design bem elaborado;
usabilidade, baseado em user experience elevada; dentre outras necessidades não funcionais. Os critérios levados em
consideração para a sua construção serão os seguintes:

\begin{itemize}
    \item Desenvolvimento rápido: por se tratar de uma simples solução, temporária, o tempo de implementação deverá
        ser curto, relevando o requisito de utilizar um framework de alta produtividade de forma escalável.
    \item Padrões de Design simples: por se tratar de uma aplicação extremamente funcional, tal que o usuário não irá
        utilizá-la por muito tempo, não há a necessidade de fortes esforços e grande elaboração dos padrões de
        User Experience.
    \item Escalabilidade baixa: como se trata de um público pequeno, para poucos usuários simultâneos, em torno de duzentos
        acessos diários, o framework pode ser de baixo desempenho, facilitando assim a sua escolha.
    \item Os usuários não carecem de executar cadastros e logins: para o escopo do piloto, cada usuário não necessitará
        fazer registros e logins no site. A política de gerenciamento dos votos será feita mediante armazenamento dos
        IP's utilizados durante o acesso, fazendo com que cada IP possa votar apenas uma vez dentro de um determinado
        intervalo de tempo.
    \item Suporte para questionários: como o piloto se trata de questionários baseados em perguntas, há a necessidade que o framework
        escolhido conceda suporte para criação de perguntas, possibilidade de votação, visualização e contagem dos votos
        separados por perguntas.
    \item Facilidade de implantação: como se trata de um site web bem rápido, há a necessidade de que este seja de fácil e rápida
        implantação, possibilitando que rapidamente seja colocado em produção. Dessa forma, o framework escolhido também carece
        de propiciar suporte para esta feature.
\end{itemize}

Dados os requisitos acima, os seguintes itens serão ponderados para a escolha da ferramenta:
\begin{itemize}
    \item Desenvolvimento Rápido
    \item Padrões de Design Simples
    \item Sem Autenticação
    \item Baixa Escalabilidade
    \item Suporte para Questionário
    \item Facilidade de Implementação
\end{itemize}

Dessa forma, o framework escolhido e estipulado para a aplicação do
piloto da About é o WordPress, que possui o desenvolvimento rápido e
muita documentação no mercado.

% web que os contemple. Para a avaliação, as ferramentas levantadas, que serão até cinco, irão ser submetidas à uma tabela que contempla
% os requisitos básicos necessários. Cada item da tabela poderá ser julgado de um a cinco, sendo que a nota 1 diz que a ferramenta não possui
% nada desta funcionalidade, já a nota 5 diz que esta contém totalemente esta funcionalidade. Assim, os valores intermediários: 2, 3 e 4 representam
% que possuem o requisito parcialmente, de acordo com a nota.
%
% % \usepackage{booktabs}
% \begin{table}[]
%     \centering
%     \begin{tabular}{@{}llllll@{}}
%         \toprule
%         \textbf{}                                                  & \textbf{Tool A}       & \textbf{Tool B}       & \textbf{Tool C}       & \textbf{Tool D}       & \textbf{Tool E}       \\ \midrule
%         \multicolumn{1}{|l|}{\textbf{Desenvolvimento Rápido}}      & \multicolumn{1}{l|}{} & \multicolumn{1}{l|}{} & \multicolumn{1}{l|}{} & \multicolumn{1}{l|}{} & \multicolumn{1}{l|}{} \\ \midrule
%         \multicolumn{1}{|l|}{\textbf{Padrões de Design Simples}}   & \multicolumn{1}{l|}{} & \multicolumn{1}{l|}{} & \multicolumn{1}{l|}{} & \multicolumn{1}{l|}{} & \multicolumn{1}{l|}{} \\ \midrule
%         \multicolumn{1}{|l|}{\textbf{Sem Autenticação}}            & \multicolumn{1}{l|}{} & \multicolumn{1}{l|}{} & \multicolumn{1}{l|}{} & \multicolumn{1}{l|}{} & \multicolumn{1}{l|}{} \\ \midrule
%         \multicolumn{1}{|l|}{\textbf{Baixa Escalabilidade}}        & \multicolumn{1}{l|}{} & \multicolumn{1}{l|}{} & \multicolumn{1}{l|}{} & \multicolumn{1}{l|}{} & \multicolumn{1}{l|}{} \\ \midrule
%         \multicolumn{1}{|l|}{\textbf{Suporte para Questionário}}   & \multicolumn{1}{l|}{} & \multicolumn{1}{l|}{} & \multicolumn{1}{l|}{} & \multicolumn{1}{l|}{} & \multicolumn{1}{l|}{} \\ \midrule
%         \multicolumn{1}{|l|}{\textbf{Facilidade de Implementação}} & \multicolumn{1}{l|}{} & \multicolumn{1}{l|}{} & \multicolumn{1}{l|}{} & \multicolumn{1}{l|}{} & \multicolumn{1}{l|}{} \\ \midrule
%                                                                    &                       &                       &                       &                       &                       \\ \midrule
%                                                                    \multicolumn{1}{|l|}{\textbf{Total}}                       & \multicolumn{1}{l|}{} & \multicolumn{1}{l|}{} & \multicolumn{1}{l|}{} & \multicolumn{1}{l|}{} & \multicolumn{1}{l|}{} \\ \bottomrule
%     \end{tabular}
%     \caption{Avaliação dos Frameworks}
%     \label{my-label}
% \end{table}

% Cada ferramenta, no final da avaliação, terá uma nota entre 6 e 30, que será disposta na linha 'Total'. Assim, a ferramenta que possuir a maior nota será escolhida para a execução do projeto piloto.
%
\subsection{Desenvolver Solução}
\label{sub:definir_tecnologia}

Dada a ferramenta escolhida, faz-se necessário que esta seja analisada em termos técnicos. Será necessário analisar e seguir alguns pontos, que
serão descritos abaixo para que a solução seja implementada com sucesso:

\begin{enumerate}
    \item Escolher versão do framework que será utilizado;
    \item Executar download do framework para o laboratório local, que terá os testes executados;
    \item Executar instalação da ferramenta em um laboratório local, que será utilizado como ambiente de
        desenvolvimento da aplicação;
    \item Definir qual template será utilizado para a página home do site, layout da aplicação e menu principal;
    \item Executar o download do plugin de execução de questionários na página principal do framework que será escolhido;
    \item Instalar na aplicação o plugin para a criação, manutenção e visualização dos questionários;
    \item Configurar plugin de questionários para armazenar as perguntas e os índices de votação de cada pergunta em persistência;
        para que posteriormente seja possível executar a análise de todos os dados coletados;
    \item Executar a criação de um questionário a fim de homologar a solução desenvolvida para os pré-requisitos estabelecidos. 
    \item Executar a integração da questão criada para homologação no layout da home da aplicação;
    \item Executar o gerenciamento de configuração de software para que o código fonte seja armazenado. Este será posteriormente
        capturado para executar a aplicação no servidor de produção;
\end{enumerate}

Com todos esses passos executados, a solução está operando de acordo como o esperado para recolher os dados básicos propostos anteriormente.
Assim, esta será preparada para ser disposta em produção em um servidor e disponibilizá-la para o público geral. Na próxima sessão serão
detalhados os passos para ter a aplicação em produção.

\subsection{Virá-la para a Produção}
\label{sub:definir_tecnologia}
Para que qualquer pessoa com acesso à internet possa conseguir ter acesso à aplicação desenvolvida, faz-se necessário que
o site hospedado esteja em um servidor com um IP externo válido. Para melhor utilização da plataforma, será necessário adquirir um domínio
que faça o apontamento para o IP do servidor adquirido. 

Os passos necessários para virar o servidor de produção serão descritos nos itens seguintes:

\begin{enumerate}
    \item Avaliar qual provedor de máquinas virtuais será utilizado;
    \item Adquirir uma máquina virtual com IP externo. Este deve conter o mínimo possível de capacidade de processamento e
        disponibilidade de memória RAM para que seja possível suportar a hospedagem da solução;
    \item Adquirir um domínio em um servidor DNS do server .com.br para apontamento do IP externo;
    \item Configurar o domínio DNS adquirido para que este execute o apontamento do IP do servidor que será utilizado;
    \item Executar instalação de um servidor de páginas HTTP no servidor;
    \item Executar a instalação de uma base de dados para armazenamento das informações obtidas;
    \item Recuperar o código fonte utilizado no ambiente de desenvolvimento para o servidor. Este será devidamente
        instalado e recuperado assim como foi feito anteriormente;
    \item Executar as configurações do framework para que este opere corretamente utilizando um servidor externo;
    \item Executar a criação novamente de uma questão para que seja possível homologar o ambiente de produção.
\end{enumerate}

Estes passos vão assegurar que o servidor seja configurado corretamente e que esteja disponível para acesso externo para
todos os usuários que vão utilizar o sistema.

Com os procedimentos necessários para que o projeto piloto esteja acessível pelos usuários, já será possível executar o marketing
para divulgar a algumas pessoas a aplicação. Os passos para o marketing serão descritos no próximo subtópico.

\subsection{Aplicar Marketing do projeto piloto}
\label{sub:definir_tecnologia}
Como se faz necessário que haja usuários utilizando o protótipo para recolher os dados, é fundamental que o propósito
e o protótipo sejam divulgados para o público externo que irá utilizá-lo. 

A proposta de marketing seguirá algumas diretivas que serão apresentadas a seguir:

\begin{enumerate}
    \item O público alvo foi definido para que este pudesse ser atingido facilmente. Como estamos tratando de um
        projeto de desenvolvendo universitário, este meio pode ser facilmente almejado em pouco tempo. Isto se
        deve ao volume de alunos existentes no campus com disponibilidade para testar novos projetos e ideias.
        Desta maneira, o público alvo serão os universitários da UnB unidade Gama - Distrito federal(UnB-FGA).
    \item Como estamos tratando de um projeto piloto que carece da presença de usuários a utilizando, alguns
        pontos são extremamente importantes para que o público alvo definido seja atingido. Assim, o meio de
        distribuição do protótipo deve conter os seguintes pontos:
        \begin{itemize}
            \item Ser possível compartilhar os links publicados no protótipo;
            \item Novas enquetes devem chegar rapidamente ao público alvo;
            \item Novas enquetes devem ser dispostas em um meio que esteja disponível para todos os
                alunos do campus FGA.
        \end{itemize}
    \item Existem vários meios possíveis para alcançar o público alvo, por exemplo:
        \begin{itemize}
            \item Contato direto verbal;
            \item Cartazes e panfletos no campus;
            \item Listas de emails;
            \item Sites de fóruns da UnB;
            \item Sites de divulgação;
            \item Grupos e páginas do facebook.
        \end{itemize}
    \item Desta forma, será escolhido o meio de comunicação que atenda de melhor maneira os pré-requisitos descritos acima. Para que
        seja possível compartilhar links, demonstrar rapidamente novas enquetes e com o maior número de estudantes, o melhor meio de
        comunicação é a utilização do facebook. Utilizando o facebook, é possível acessar o grupo da faculdade, que contém vários alunos,
        levando em consideração que os links podem ser compartilhados por lá. Dessa forma, toda a apresentação do protótipo será executada
        via grupo da faculdade no facebook;
    \item Todas as novas enquetes serão apresentadas no grupo e compartilhadas também no grupo do facebook do campus da faculdade.
\end{enumerate}

Assim, todas as novas enquetes irão seguir os padrões estabelecidos nestas diretrizes de regras de marketing. Isto irá assegurar que
o público alvo escolhido seja almejado.


\subsection{Manter a Solução}
\label{sub:definir_tecnologia}
Após a construção estabelecida e disponível para que os usuários a utilizem, já é possível aplicar novos questionários, fazendo uso do plano de marketing.
Esta etapa consistirá em criar um sistema em que os próprios usuários vão ceder as novas informações para novos questionários. Estes questionários 
terão as informações coletadas e futuramente utilizadas.

Primeiramente, será criada uma segunda página para o site. Esta página será responsável por conter uma enquete com a seguinte pergunta:

 \begin{quote}
     "Qual deve ser a próxima enquete do site?"
 \end{quote}

As perguntas dispostas serão analisadas e as que foram consideradas de bom gosto, serão utilizadas. A cada dia, uma nova questão será apresentada
para a enquete. Essa nova enquete será publicada e compartilhada. Após 48 horas, esta será dada como finalizada e o resultado será apresentado
para o público. 

Este ciclo será mantido por duas semanas, possibilitando com que sejam elaborados materiais suficientes para recolher os indicadores que necessitamos 
das técnicas de gamificação da rede social.
\subsection{Finalizar a Solução}
\label{sub:definir_tecnologia}

Após duas semanas de uso da solução proposta, os dados serão recolhidos e o servidor será desligado para evitar gastos. O domínio continuará em operação
por mais um ano, porém, não apontará para um endereço de IP válido, pois, o servidor que conterá o endereço externo não estará mais em operação.

Este será o tempo necessário para implantar e recolher todas as informações propostas para o uso da solução de projeto piloto.

\section{Levantamento das Técnicas de Gamificação}
\label{sub:survey_para_t_cnicas_de_gamifica_o}
Será executado um levantamento com alguns usuários aleatoriamente escolhidos dentre os que interagiram com o piloto.
O objetivo do levantamento é identificar quais são as técnicas de gamificação mais presentes no piloto executado.

O levantamento será executado em duas etapas. A primeira se trata de conseguir entender o que os usuários entendem e pensam sobre o objetivo
principal que o projeto piloto terá a intenção de retratar. A segunda parte consistirá na elaboração de um survey com opções de valores entre 1 e 5,
listando todas as técnicas de gamificação existentes no octalysis.

Os procedimentos sobre como serão elaboradas as duas próximas etapas serão descritos nas duas sessões seguintes. 

\subsection{Características do Projeto Piloto}
\label{sub:caracter_sticas_projeto_do_piloto}
Com a intenção de compreender a visão que os usuários irão ter do projeto, bem como entender onde podem ser trabalhadas suas motivações básicas,
serão levantadas as suas características.

Este processo será elaborado fazendo com que os usuários respondam três perguntas abertas. As perguntas são as seguintes:

\begin{quotation}
    Questão 01: Na sua opinião, o que este site representava?
\end{quotation}

\begin{quotation}
    Questão 02: O que você acredita que motivava e levava as pessoas a utilizarem o site?
\end{quotation}

\begin{quotation}
    Questão 03: E quanto ao contrário, o que você acredita que levava as pessoas a se desmotivarem e a não
    utilizarem mais o site?
\end{quotation}

Estas questões serão feitas a alguns usuários do sistema individualmente. As suas respostas serão gravadas para futuras análises.
Além disso, as respostas serão transcrevidas para o relatório.

\section{Survey das Técnicas}
\label{sub:survey_das_t_cnicas}
Com o intuito de analisar as motivações básicas em questão de quanto é necessário para que esta se enquadre no projeto
de gamificação, iremos aplicar um survey com várias técnicas oriundas do octalysis. Essas técnicas serão submetidas à alguns
usuários do protótipo para avaliarem o quanto aquela técnica se enquadra dentro do escopo. Essas técnicas serão julgadas de 1
a 5. Sendo que da mesma forma para que a avaliação das ferramentas e frameworks, serão atribuídas às técnicas com
muita presença a nota cinco e para as técnicas com pouca presença a nota um. Sendo assim, os valores intermediários
também representam presenças intermediárias destes, na mesma proporção.

As técnicas que serão aplicadas já foram selecionadas no framework. Elas serão listadas a seguir:

\begin{multicols}{2}
    \begin{enumerate}
        \item Narrative
        \item Beginner's Luck
        \item Free Lunch
        \item Elitism
        \item Humanity Hero
        \item Higher Meaning
        \item Destiny Child
        \item Status Points/Points
        \item Achievement Symbols/ Badges
        \item Leaderboards
        \item Progress Bar
        \item Glowing Choice
        \item Desert Oasis
        \item The Rockstar Effect
        \item Fixed Actions Rewards/Earned Lunch
        \item Quest List
        \item High Five
        \item Crowning
        \item LevelUp Symphony
        \item Aura Effect
        \item Step-by-Step Tutorial
        \item Boss Fight
        \item Evergreen Mechanics
        \item General's Carrot
        \item Real-Time Control
        \item Chain Combos
        \item Instant Feedback
        \item Blank Fills
        \item Voluntary Autonomy
        \item Choice Perception/Poison Picker
        \item Plant Picker/ Meaningful Choices
        \item Milestone Unlock
        \item Boosters
        \item Virtual Goods
        \item Build from Scratch
        \item Collection Set
        \item Exchangeable Points
        \item Monitor Attachment
        \item The Alfred Effect
        \item Mentorship
        \item Bragbuttons
        \item Trophyshelves
        \item Group Quest
        \item Social Treasures
        \item Social Proud
        \item Conformy Anchor
        \item Water Cooler
        \item Friending
        \item SeeSaw Bump
        \item Touting
        \item Bragging
        \item Thank-You Economy
        \item Dangling
        \item Ancored Juxtaposition
        \item Prize Pacing
        \item Options Pacing
        \item Patient Feedback
        \item Count Down
        \item Throttles
        \item Moats
        \item TortureBreak
        \item Envolved UI
        \item Glowing Choice
        \item Mystery Boxes/ Random Rewards
        \item Easter Eggs
        \item Sudden Rewards
        \item Visual Storytelling
        \item Obvious Wonder
        \item Rolling Rewards
        \item Mischief Puzzle
        \item Oracle Effect
        \item Lottery (Rolling Rewards)
        \item Mini Quests
        \item Countdown Timer
        \item Status Quo Sloth
        \item FOMO Ponch
        \item Sunk-Cost Tragedy
        \item Lost Progress
        \item Scarlett Letter
        \item Visual Grave
    \end{enumerate}
\end{multicols}

Agora, como será mapeado o quanto cada motivação básica está presente no projeto piloto? Simples, como dito antes,
cada técnica pertence a uma motivação básica. Desta forma, será simples mapear o quanto cada motivação estará presente.

Como citado anteriormente, existem oito motivações básicas, e cada uma engloba um conjunto de técnicas. As oito motivações básicas são as seguintes:

\begin{enumerate}
    \item Significado Épico \& Chamado
    \item Desenvolvimento \& Realização
    \item Empoderamento \& Feedback
    \item Propriedade \& Posse
    \item Influência Social \& Pertencimento
    \item Escassez \& Impaciência
    \item Imprevisibilidade \& Curiosidade
    \item Perda \& Rejeição
\end{enumerate}

Como se dará o cálculo para verificar a presença de uma motivação básica na aplicação?

Uma média simples poderia atender a identificação das motivações básicas, dada pela seguinte fórmula: \\ \\


$ MediaCadaMotivacao = \frac{PontosTotais}{QuantidadeMotivacoes} $

$ PontosMotivacao = \frac{MediaCadaMotivacao}{\sum PontosCadaMotivacao} $


$Onde:
\\ \\$
QuantidadeMotivacoes = 8
\\ PontosTotais = Somatório do máximo de pontos possíveis dentro da votação.
\\ MediaCadaMotivacao = Média aritmética de quantos pontos máximos cada motivação pode conter.
\\ PontosCadaMotivacao = Quantidade de pontos obtida pela votação.
\\ PontosMotivacao = Percentagem de pontos que a motivação possui. \\ \\


 Porém, observamos que esta média se torna injusta com algumas motivações e compromete o resultado final.
Isso ocorre pois  algumas motivações básicas possuem mais técnicas do que outras, fazendo com que
as que possuem poucas técnicas consigam somar uma baixa quantidade de pontos para aplicar no percentual final perante
à média de todas as outras.

Desta forma, para que as médias sejam relativas dentro de cada motivação, foi decidido que as médias seriam avaliadas dentro da
própria motivação. Desta forma, será executado um cálculo para cada motivação, onde o objetivo é identificar quais são
as suas respectivas notas máximas possíveis. Logo após, serão executadas contagens dos pontos de cada motivação quanto à votação.
Esses dados serão utilizados para resultar a média, dividindo os pontos totais pelos pontos de votação. 

Desta forma, a equação para calcular a média de cada pontuação se dará como o descrito a seguir:

$PontosMotivação = \frac{\sum PontoMáximo}{PontosCadaMotivacao}\\Onde:$

PontoMáximo = Quantidade máxima de ponto dentro de cada técnica.


Desta forma, teremos total certeza que cada motivação básica terá a sua nota justa perante todas as demais e não será prejudicada
pela sua quantidade. Assim, será possível aplicar estas fórmulas para obter os dados corretamente.

Através dos dados utilizando as médias, será possível desenhar um primeiro esboço do framework, identificando quais são as motivações
básicas que os usuários mais conseguem ver na aplicação. 

O criador do Octalasys implementou e disponibilizou uma ferramenta para executar desenhos da forma como o framework está implementado
de acordo com o quão presentes são algumas técnicas. Esta aplicação pode ser vista no site oficial do framework.

Abaixo está ilustrada um exemplo da figura do framework,  que possui uma ferramenta para executar desenhos da forma como o framework está implementado
de acordo com o quão presentes são algumas técnicas. Esta aplicação pode ser vista no site oficial do framework.

A Figura \ref{fig:exoctalysis} está ilustrada um exemplo da figura do framework.

 \begin{figure}[h]
     \centering

     \includegraphics[width=300px, scale=1]{figuras/exoctalysis}
     \caption{Exemplo do framework octalysis}

     \label{fig:exoctalysis}
 \end{figure}


Nela, podemos observar que cada motivação básica possui algum valor atribuído, e isto representa o quanto desta existe no escopo que
está sendo testado. O percentual de quanto cada motivação básica é importante estará representado na Tabela \ref{exmotivacao}:

% Please add the following required packages to your document preamble:
% \usepackage{booktabs}
\begin{table}[]
    \centering
    \caption{Índice de presença das motivações: exemplo}
    \label{exmotivacao}
    \begin{tabular}{@{}|c|c|@{}}
        \toprule
        \textbf{Motivação Básica}          & \textbf{\begin{tabular}[c]{@{}c@{}}Porcentagem\\ (\%)\end{tabular}} \\ \midrule
            Significado Épico \& Chamado       & 100                                                                 \\ \midrule
        Desenvolvimento \& Realização      & 60                                                                  \\ \midrule
        Empoderamento \& Feedback          & 40                                                                  \\ \midrule
        Propriedade \& Posse               & 90                                                                  \\ \midrule
        Influência Social \& Pertencimento & 40                                                                  \\ \midrule
        Escassez \& Impaciência            & 70                                                                  \\ \midrule
        Imprevisibilidade \& Curiosidade   & 30                                                                  \\ \midrule
        Perda \& Rejeição                  & 30                                                                  \\ \bottomrule
    \end{tabular}
\end{table}


Desta forma, será necessário gerar este framework para as médias básicas retiradas do projeto piloto e desenhar esta tabela com os
seus respectivos valores, que, provavelmente não serão tão precisos quanto aos que foram utilizados no exemplo.

A figura do framework e a tabela que serão gerados permitirão à quem irá analisar e avaliar o framework ter uma interpretação
visual dos níveis de motivações básicas no sistema e como elas se relacionam.

Utilizando esses cálculos básicos de média, e com as notas da votação, será possível realizar cálculos estatísticos para a construção
das análises a respeito das técnicas de gamificação.



\section{Análise Estatística das Técnicas}
\label{sub:an_lise_estat_stica_das_t_cnicas}
A partir da massa de dados obtida com o survey, serão executadas análises estatísticas a fim de identificar correlações
entre as regras e permitir que seja possível entender quais são as técnicas mais presentes e o mais importante, como
estas se relacionam entre si, para que seja possível fazer aferições sobre qual técnica está ligada a qual outra.

Para realizar estas análises estatísticas, são necessários alguns passos que serão justificados a seguir:

\begin{enumerate}
    \item Definir ferramenta para realização das análises estatísticas: a massa de dados é consideravelmente grande, devido 
        a grande quantidade de técnicas valoradas. Assim, utilizar planilhas do excel ou simplesmente cálculos diretos
        podem dificultar a utilização, realização e armazenamento de cálculos. Dessa forma, será adotada uma linguagem
        de script utilizada para análises estatísticas;
    \item Importação dos dados oriundos do survey;
    \item Parser dos dados importados;
    \item Calcular o alpha de cronbach, para averiguar e conseguir estabelecer a confiabilidade do survey analisado;
    \item Para identificar as relações entre as técnicas, primeiramente, será utilizada a correlação de pearson;
    \item Sumarizar das correlações de pearson;
    \item Transformar dos dados da matriz em transações;
    \item Calcular o algoritmo apriori, para identificar quais técnicas irão ser diretamente ligadas as outras.
\end{enumerate}

Desta forma, nas sessões seguintes, serão detalhados os passos que foram agora pouco pontuados.

\subsection{Ferramenta para Estatística}
\label{sub:ferramenta_para_estat_stica}
Como foi dito agora pouco, para o escopo que estamos trabalhando, apenas uma planilha excel não será suficiente
para a manutenção, processamento e armazenamento dos dados obtidos. Dessa forma, será  necessário um processo
para levantamento da ferramenta que será utilizada.

Nosso escopo irá carecer de alguns requisitos que serão descritos a seguir:

\begin{itemize}
    \item Suporte à importação de dados em arquivos ods, xls, csv;
    \item Suporte à exportação de dados em arquivos ods, xls, csv;
    \item Suporte a algoritmo de cálculo da correlação de pearson;
    \item Suporte a algoritmo de cálculo do alpha de cronbach;
    \item Suporte a algoritmo de relações apriori;
    \item Possibilidade de manipulação de matrizes;
    \item Possibilidade para executar os mesmos passos consecutivas vezes;
    \item Flexibilidade para manutenção.
\end{itemize}

Para tanto, foi executada uma pesquisa sobre as ferramentas de análise estatística
disponíveis e verificou-se que a que mais se enquadra nos padrões solicitados
é a Rscript, contento todos os pontos e requisitos requeridos para a
avaliação dos dados.

% Para tanto, serão levantadas três ferramentas de análise estatística. Estas serão submetidas a uma ficha de avaliação.
% Cada requisito terá uma nota de um a cinco para itens que podem ser valorados, sendo que a ferramenta que melhor pontuar será a escolhida.
%
% A Tabela \ref{avalicaoferramentaestatistica} será utilizada para a avaliação de atributos, sendo que serão
% pontuados entre um e cinco. Sendo que se não possuir o atributo, será adotada a nota 1, e caso possua, será
% atribuída a nota cinco.
% % Please add the following required packages to your document preamble:
% % \usepackage{booktabs}
% \begin{table}[]
%     \centering
%     \caption{Avaliação - Ferramenta Estatística}
%     \label{avalicaoferramentaestatistica}
%     \begin{tabular}{@{}llll@{}}
%         \toprule
%         \multicolumn{1}{|l|}{}                                       & \multicolumn{1}{l|}{\textbf{Ferramenta A}} & \multicolumn{1}{l|}{\textbf{Ferramenta B}} & \multicolumn{1}{l|}{\textbf{Ferramenta C}} \\ \midrule
%         \multicolumn{1}{|l|}{\textbf{Manipulação de matrizes}}       & \multicolumn{1}{l|}{}                      & \multicolumn{1}{l|}{}                      & \multicolumn{1}{l|}{}                      \\ \midrule
%         \multicolumn{1}{|l|}{\textbf{Comandos repetidas vezes}}      & \multicolumn{1}{l|}{}                      & \multicolumn{1}{l|}{}                      & \multicolumn{1}{l|}{}                      \\ \midrule
%         \multicolumn{1}{|l|}{\textbf{Flexibilidade de manutenção}}   & \multicolumn{1}{l|}{}                      & \multicolumn{1}{l|}{}                      & \multicolumn{1}{l|}{}                      \\ \midrule
%         \multicolumn{1}{|l|}{\textbf{Importação CSV}}                & \multicolumn{1}{l|}{}                      & \multicolumn{1}{l|}{}                      & \multicolumn{1}{l|}{}                      \\ \midrule
%         \multicolumn{1}{|l|}{\textbf{Importação XLS}}                & \multicolumn{1}{l|}{}                      & \multicolumn{1}{l|}{}                      & \multicolumn{1}{l|}{}                      \\ \midrule
%         \multicolumn{1}{|l|}{\textbf{Importação ODS}}                & \multicolumn{1}{l|}{}                      & \multicolumn{1}{l|}{}                      & \multicolumn{1}{l|}{}                      \\ \midrule
%         \multicolumn{1}{|l|}{\textbf{Exportação CSV}}                & \multicolumn{1}{l|}{}                      & \multicolumn{1}{l|}{}                      & \multicolumn{1}{l|}{}                      \\ \midrule
%         \multicolumn{1}{|l|}{\textbf{Exportação XLS}}                & \multicolumn{1}{l|}{}                      & \multicolumn{1}{l|}{}                      & \multicolumn{1}{l|}{}                      \\ \midrule
%         \multicolumn{1}{|l|}{\textbf{Exportação ODS}}                & \multicolumn{1}{l|}{}                      & \multicolumn{1}{l|}{}                      & \multicolumn{1}{l|}{}                      \\ \midrule
%         \multicolumn{1}{|l|}{\textbf{Alpha de Croncach}}             & \multicolumn{1}{l|}{}                      & \multicolumn{1}{l|}{}                      & \multicolumn{1}{l|}{}                      \\ \midrule
%         \multicolumn{1}{|l|}{\textbf{Algorítimo Apriori}}            & \multicolumn{1}{l|}{}                      & \multicolumn{1}{l|}{}                      & \multicolumn{1}{l|}{}                      \\ \midrule
%         \multicolumn{1}{|l|}{\textbf{Transformação para transações}} & \multicolumn{1}{l|}{}                      & \multicolumn{1}{l|}{}                      & \multicolumn{1}{l|}{}                      \\ \midrule
%         \multicolumn{1}{|l|}{\textbf{Correlação de Pearson}}         & \multicolumn{1}{l|}{}                      & \multicolumn{1}{l|}{}                      & \multicolumn{1}{l|}{}                      \\ \midrule
%                                                              &                                            &                                            &                                            \\ \midrule
%         \multicolumn{1}{|l|}{Total}                                  & \multicolumn{1}{l|}{}                      & \multicolumn{1}{l|}{}                      & \multicolumn{1}{l|}{}                      \\ \bottomrule
%     \end{tabular}
% \end{table}
%
% Após totalizar as pontuações totais das ferramentas, está será selecionada e preparada para a utilização e execução dos cálculos.

Os passos necessários para que a ferramenta esteja apta para o uso das análises dos scripts serão detalhados abaixo:

\begin{enumerate}
    \item Download da ferramenta através do site oficial;
    \item Instalação da ferramenta no ambiente local que será desenvolvido o código;
    \item Adaptação da IDE utilizada pelo desenvolvedor para que seja compatível à ferramenta escolhida.
\end{enumerate}

Adotados estes passos, a ferramenta estará pronta para receber a aplicação e executar todos os scripts e cálculos que serão
necessários.

\subsection{Importação dos Dados}
\label{sub:importa_o_dos_dados}
De antemão, já se tem informação que  os dados do survey serão recolhidos e armazenados em uma planilha de algum dos seguintes tipos:

\begin{itemize}
    \item Planilhas Excel do Word - XLS;
    \item Planilhas Calc do Libre - ODS;
    \item Planilhas Separada por Vírbula - CSV.
\end{itemize}

Assim, já sabemos que será necessário executar a importação destes dados. Como a ferramenta escolhida tem como pré-requisitos
essa feature, esta não será uma preocupação, pois sabemos que ela se encarregará da execução deste ponto. Porém, para realizar
a importação corretamente, serão necessários alguns passos:

\begin{enumerate}
    \item Executar a instalação dos pacotes necessários para ler os tipos de arquivos citados anteriormente;
    \item Carregar o pacote a ser utilizado para a importação que ocorrerá;
    \item Executar a importação dos dados através do arquivo de armazenamento da planilha;
    \item Armazenar em memória RAM os dados importados.
\end{enumerate}

\subsection{Alpha de Cronbach}
\label{sub:alpha_de_cronbach}
Como é possível identificar se os resultados do nosso survey está confiável?

É possível que alguma variável esteja diminuindo o índice de confiabilidade do questionário?

Essas perguntas são facilmente respondidas pelo alpha de cronbach. Este irá identificar entre as regras quais possuem
um elevado nível de correlação entre si, aferindo através de suas fórmulas qual a confiabilidade do que foi obtido.

O Alpha retorna valores entre um e zero, onde os valores aceitáveis estão na faixa de 0.6 e 0.9.

Também é possível que o Alpha nos retorne qual seria o novo índice de confiabilidade caso um item específico seja removido.
Isso permite com que isolando algumas partes do questionário, a confiabilidade melhore.

Alguns passos são necessários para calcular o Alpha. Estes procedimentos serão descritos a seguir:

\begin{enumerate}
    \item Instalar o módulo para o Alpha de Cronbach;
    \item Carregar o módulo para o Alpha de Cronbach;
    \item Remover todos os campos dos dados com valores nulos;
    \item Remover todos os campos da tabela com números diferentes de [1, 2, 3, 4, 5];
    \item Habilitar a aplicação para apresentar os cabeçalhos das variáveis;
    \item Executar a função que executa o cálculo do Alpha;
    \item Averiguar se o resultado do índice está dentro dos padrões aceitáveis para o Alpha de Cronbach;
    \item Caso o valor esteja fora dos valores permitidos, desabilite a variável que mais está tendo impacto negativo no indicador,
        ou seja, remova aquela com o índice mais distante de 0.6 e 0.9, e volte para o item 6.
    \item Caso o valor esteja satisfatório, passe a considerar para os próximos cálculos apenas as variáveis que foram
        escolhidas com alta confiabilidade;
    \item Para que seja possível utilizar um tracking dos resultados posteriormente, estes devem ser exportados para uma planilha.
        Esta planilha deve ser de algum dos tipos estipulados para a escolha da ferramenta, de modo a garantir que esta terá suporte
        para efetuar uma dada atividade.
    \item Para melhor organização do projeto, deverá ser criada uma pasta que armazenará todos os resultados gerados pelo cálculo do Alpha
        de Cronbach.
\end{enumerate}

Agora que é possível ter confiabilidade nos dados, de acordo com os resultados apresentados pelos algoritmos utilizados, será
possível
utilizar outros cálculos para aferir a correlação entre as técnicas.

O primeiro ponto requerido é, a partir dos dados tratados, aplicar a correlação de pearson. Esta correlação irá ilustrar
quanto cada técnica se assemelha com outra. Esta é representada em uma matriz, contendo em cada célula, um valor que varia entre um
e zero dizendo o quão estas variáveis são semelhantes. As que possuem correção 1 entre si, são totalmente semelhantes, assim como as que tem
correlação 0 são totalmente distintas e assim por diante.

Para aferir estes valores será necessário seguir os seguintes passos:

\begin{enumerate}
    \item Instalar o módulo para a Correlação de Pearson;
    \item Carregar o módulo  na aplicação para a Correlação de Pearson;
    \item Executar a função de cálculo de correlação utilizando a matriz com dados confiáveis, obtidos a partir do alpha de cronbach.
    \item Armazenar em memória o resultado do Alpha
\end{enumerate}

Após estes passos, será possível armazenar em memória as correlações e resgatá-las, será possível fazer uma sumarização dos dados
da correlação. Com qual objetivo estes passos serão executados?

Simples, adaptar e melhorar a visualização dos dados. E, por fim, gerar uma tabela onde seja possível identificar o quanto uma técnica
se correlaciona com todas as demais. Isto propiciará com que seja possível identificar quais técnicas tem maior índice de correlação
dentre todas as demais, o que proporcionará ter o poder de descobrir qual técnica é mais influente e se relaciona com todas as demais.


Para executar essa sumarização, são necessários alguns passos que serão descritos a seguir:

\begin{itemize}
    \item Sumarização dos dados da correlação;
    \item Substituir o quanto cada técnica se relaciona por um valor numérico;
    \item Efetuar o somatório de pontos de uma data técnica abordada;
    \item Identificar quantos pontos existem no total;
    \item Executar o cálculo da média para cada técnica, a fim de identificar o quanto, em porcentagem, cada uma
        se relaciona com as demais;
    \item Ordenar todas as técnicas a fim de identificar quais mais se relacionam com todas as demais.
\end{itemize}

Desta forma, todos os dados relativos à correlação de pearson já estarão identificados. Estes devem ser armazenados em persistência,
em uma tabela de um dos tipos suportados pela ferramenta. Desta forma, agora será possível executar os procedimenos para aplicar
o algoritmo apriori.

O algoritmo apriori irá possibilitar que seja possível visualizar algumas regras, com uma confiabilidade estabelecida,
quanto às técnicas a nível de:
\begin{itemize}
    \item Toda vez em que a técnica X ocorrer, a Y também irá ocorrer.
    \item Toda vez que a técnica Y e Z ocorrerem, a W também irá ocorrer.
\end{itemize}

Desta forma, alguns passos devem ser executados para que o apriori seja corretamente calculado:

\begin{enumerate}
    \item Instalar o pacote do apriori na máquina utilizada para os testes;
    \item Carregar o módulo do apriori na máquina que será utilizada;
    \item Transformar todos os valores da tabela de dados em 0 e 1, fazendo com que as notas abaixo de três
        sejam transformadas para um e as maiores, para cinco;
    \item Executar a transformação das células da tabela de dados em transações;
    \item Executar o algoritmo apriori e armazenar o resultado em memória RAM;
    \item Sumarizar os dados recebidos do algoritmo e armazená-los em uma planilha com o suporte
        estabelecido.
\end{enumerate}

Este resultado irá permitir com que seja possível identificar as regras de existência das técnicas. Assim, será possível fazer
a ligação sobre: quais regras sempre acontecem quando as duas técnicas que mais tem correlação com as demais, a partir de pearson,
por exemplo, também acontecem.


Desta forma, será possível identificar quais as técnicas também se relacionam com aquelas que mais estão presentes no escopo do
projeto.

Voltando novamente na correlação de pearson, será possível observar o quanto essas técnicas secundárias se relacionam com as demais. 
Dessa forma, é possível atribuir notas para cada uma. Essas notas serão relacionadas com suas respectivas motivações básicas e somadas, e assim
será possível obter um framework.

Estes detalhes para a elaboração no novo framework será discutida na próxima sessão \ref{sub:constru_o_do_framework}

\section{Construção do Framework}
\label{sub:constru_o_do_framework}
Utilizando as análises estatísticas realizadas com base no survey, serão extraídos os dados de quais técnicas de gamificação
devem ser mais presentes na RSA.

O objetivo é  utilizar técnicas que permitam que haja uma forte correlação entre si.

Após os dados das correlações de pearson e apriori, conseguimos identificar, como dito na sessão passada, quais são as técnicas
que sempre acontecem quando alguma das técnicas que mais tem correlação entre as demais também acontecem.

Assim, serão atribuídos pontos para cada motivação básica de acordo com a soma dos pontos das técnicas primárias, obtidas pelo ranking da
tabela da correlação de pearson, e das técnicas secundárias, obtidas pelo algoritmo apriori, como foi dito anteriormente.

Dessa forma, com os pontos obtidos de cada motivação, serão levantados pontos e valores, possibilitando a criação de um novo framework,
baseado nas principais motivações e técnicas que devem ser presentes no framework.

\section{Objeto de Gamificação}
\label{sub:objeto_de_gamifica_o}
Dadas as técnicas escolhidas, será analisado pelo proprietário do produto qual é o melhor objeto a ser gamificado na RSA.

Este objeto será alvo das técnicas e das implementações para desenvolver as motivações básicas necessárias.


\section{Implementação das Técnicas}
\label{sub:implementa_o_das_t_cnicas}
A partir do objeto escolhido, será possível implementar o código que fará a RSA ser gamificada.

Será criado um módulo na aplicação responsável por gerir, apresentar, interagir e relatar análises
dos componentes de gamificação que serão executados. Mas como são estes módulos? Quais as restrições existentes?

A solução destes pontos serão definidas e esclarecidas a seguir:

\begin{itemize}
    \item A About será construída em Python 3.5;
    \item  O framework web utilizado para o desenvolvimento será o Django;
    \item A implementação da rede social em si não faz parte do escopo do trabalho, porém, a implementação
        das técnicas na rede social, sim;
    \item O framework será modularizado, por isto, o desenvolvimento do módulo será realizado em um app
        Django, sendo passível de reutilização.
\end{itemize}

Para a execução da rede social, o critério utilizado será a familiaridade do desenvolvedor para com a tecnologia
utilizada. Desta forma, o  ponto mais conhecido pelo desenvolvedor em questão é o Python 3.5. Assim, este será adotado.

O django será utilizado, pois como a RSA irá utilizar vários módulos e ferramentas, uma ferramenta  bastante completa
se faz mais útil do que o flask, que e reduzido é pequeno. Desta forma, essa será a escolha para que o desenvolvimento seja
realizado com êxito.

Como estamos tratando de um código extenso, será necessário manter a manutenabilidade deste, possibilitando a reutilização dos
módulos. Desta forma, será desenvolvido um App django que conterá todas as diretivas necessárias para implementar os módulos, que serão aplicados
nos objetos de gamificação.

Por fim, para suportar tudo o que será elaborado no código, com boas libs implementadas e bem documentadas, será o framework citado
acima. Isto garante que a produtividade de desenvolvimento do projeto seja elevada. Dessa forma, o App utilizado também deverá ser
desenvolvida na linguagem e no framework citados acima, para que haja compatibilidade.

Para tanto, se faz necessário que para orquestrar o desenvolvimento, será necessário adotar uma metodologia de processos. Utilizando 
assim suas tecnologias e boas práticas.

\subsection{Método de Desenvolvimento}
\label{sub:m_todo_de_desenvolvimento}
Assim, será adotado um método de avaliação de círculos de desenvolvimento, para que seja possível escolher o melhor.
O ciclo de desenvolvimento adotado que será aplicado deve ter os seguintes fatores:

\begin{enumerate}
    \item Devido a ter poucos envolvidos no processo, o método deve ser rápido, interativo;
    \item Como o orientador, desenvolvedor e cliente estão próximos, a interação deve ser alta;
    \item Além do relatório, não são necessários muitos documentos para a formalização do desenvolvimento;
    \item Ferramentas existentes para dar suporte ao desenvolvimento, de maneira rápida e bem visível para a equipe;
    \item O aprendizado deve circular rapidamente entre os envolvidos;
    \item Reuniões presenciais devem ocorrer em um período de até uma semana.
\end{enumerate}

Dessa forma, será definido um processo de desenvolvimento que seja adaptável
ao projeto. Assim, a metodologia a ser definida deve conter os seguintes pontos:

\begin{itemize}
        \item Poucos  envolvidos no projeto;
        \item Equipe próxima;
        \item Baixa necessidade de documentação dos processos;
        \item Existência de ferramentas de suporte rápido;
        \item Suporte para conhecimento compartilhado;
        \item Curto prazo entre reuniões presenciais.
\end{itemize}

Dessa forma, para se enquadrar nos requisitos do projeto, a metodologia de desenvolvimento
adotada será a adaptativa, com aplicação do Scrum e sprints definidas.

\subsection{Time Scrum}
\label{subsec:timescrum}
O time Scrum será constituído por três papeis: Product Owner, Scrum Master e Development Team.

Esses papeis serão desempenhados por duas pessoas. O orientando Tiago R. Assunção será o Product
Owner, bem ocmo o Development Team.

O papel de Scrum Master será desempenhado pelo orientador Dr. Sergio. 

\subsection{Eventos da Metodologia}
\label{subsec:eventosdametodologia}
Serão utilizados os eventos que são definidos na metodologia Scrum. Estes serão aplicados
durante o processo de desenvolvimento tal qual será descrito nos seguintes subcapítulos.
O primeiro é a Sprint, que será definida a seguir.
O segundo diz respeito à Sprint Planning. A terceira parte irá tratar sobre a Sprint Retrospective.
E por fim, será apresentada a Sprint Review.

\subsubsection{Sprint}
\label{subsec:sprint}
A Sprint do projeto será definida de acordo com um tempo suficiente para que o Product Owner e o 
Scrum Master consigam acompanhar corretamente o projeto e que seja possível avaliar o quanto
será executado.

Dessa forma, como os interessados no projeto possuem restrições de tempo e o projeto pode
ser implementado com reuniões pouco espaçadas, será adotado o tempo para a Sprint de duas
semanas.


\subsubsection{Sprint Planning Meeting}
\label{subsec:sprintplanningmeeting}
Esta será executada ao início de toda a Sprint, para que seja possível planejar
todas as histórias que serão executadas ao longo das duas semanas. 

\subsubsection{Sprint Retrospective}
\label{subsec:sprintplanningmeeting}
Para a retrospectiva, será executado um balanço sobre como foram elaborados e qual o resultado
da sprint que acabou de terminar. Este evento será executado imediatamente após à
revisão da Sprint. 

Será executada com a presença de todos os papeis definidos no Scrum.


\subsubsection{Sprint Review}
\label{subsec:sprintreview}
Esta será responsável por apresentar todos os entregáveis que foram planejados para a presente
Sprint. Onde será apresentado oa Product Owner, Scrum Master e Development Team.

Será executada ao final do último dia definido na Sprint, dando início ao processo dos eventos
Scrum pré e pós Sprint.

\subsection{Scrum Artifacts}
\label{subsec:scrumartifacts}
Dentre todos os artefatos Scrum estabalecidos, serão utilizados dois, que serão descritos nas
subsessões a seguir: Product Backlog e Sprint Backlog.

\subsubsection{Product Backlog}
\label{subsec:productbacklog}
Os requisitos gerais serão pautados nas necessidades da rede social, que por sua vez, serão
baseadas nos objetivos traçados para implementar o projeto de gamificação na RSA. 

O Product Backlog irá servir para regir todas as atividades que serão elaboradas ao longo do desenvolvimento. 
Os pontos definidos para serem elaborados são os seguintes a fim de promover
com que o usuário execute as seguintes ações, definidos como requisitos:

\begin{itemize}
    \item Conhecer a Rede Social About;
    \item Clicar no link da Rede Social About
    \item Conhecer as features oferecidas pela Rede Social
    \item Executar o tutorial de uso da About;
    \item Compartilhar a Rede Social About com os amigos
    \item Adicionar foto e email na network
    \item Permitir a inscrição na lista de email
    \item Fazer login diariamente na network
    \item Abrir semanalmente os emails enviados pela network
    \item Compartilhar abouts com os amigos
    \item Participar de grupos no facebook sobre a rede social about
    \item Adquirir a versão prêmio da rede social about
    \item Inscrever em grupos de discussão sobre a rede social about
    \item Escrever mais de um about diariamente
    \item Votar em mais de vinte abouts diarios
    \item Se tornar contribuidor da Rede Social About;
    \item Fazer parte da equipe de desenvolvedores da About
    \item Propor melhorias para a about
    \item Tornar-se moderador dos abouts
\end{itemize}


\subsubsection{Sprint Backlog}
\label{subsec:sprintbacklog}
Será construído em toda a Sprint Planning, avaliando quais histórias irão
sair do Product Backlog para integrar o Sprint Backlog.

Este será utilizado para focar o desenvolvimento das histórias ao longo de
toda a Sprint.
