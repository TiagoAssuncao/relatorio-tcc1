\section{\textit{Octalysis} Framework}
\label{sub:octalysisframework}
O \textit{Framework} \textit{Octalysis} foi desenvolvido pelo \cite{chou2015actionable} com a intenção de ser uma
base de auxílio para pessoas sem nenhuma capacitação sobre gamifição
conseguir definir e aplicar este processo em uma base qualquer.

Este Framework criado por \cite{chou2015actionable} possui várias mecânicas de funcionamento,
bem quanto à sua divisão tanto quanto à sua maneira de lidar com as
motivações básicas de casa usuário. Dessa forma, o subcapítulo a seguir
apresenta a forma com que o \textit{Octalysis} trabalha e quais são suas divisões básicas.

O Segundo subcapítulo diz respeito às fases do \textit{Octalysis}, bem como o último
apresentará uma estratégia de aplicação no cenário, já definida pelo próprio
\cite{chou2015actionable}.


\subsection{Mecânica do \textit{Octalysis}}
\label{sub:mecanicaoctalysis}
O funcionamento do Framework \textit{Octalysis} é subdividido em três partes, onde,
cada uma representa sua função específica. Essas três partes estão descritas
nas subsessões a seguir. A primeira trata sobre as definições de Motivações
Básicas, onde estas são dívidas em oito e cada uma tem sua função específica.
A segunda trata sobre a divisão direita e esquerda do Framework, onde cada
lado representa uma forma diferente de sentir diante certa ação.
A terceira e última trata sobre as definições de emoções, onde existem
pensamentos e sentimentos bons e por outro lado, sentimentos ruins.
Dessa forma, a seguir estão as próximas sessões.

\subsubsection{Oito Motivações Básicas}
\label{sub:oitomotivacoesbasicas}
As motivações básicas são ações que levam, motivam e engajam
o usuário para que este execute uma dada atividade alvo.
A diferença entre estas é que cada uma foca um tipo de sentimento
do usuário. Essas motivações contém um grupo de técnicas de \textit{game}
que podem ser aplicadas em um dado contexto para alcançar algum
objetivo. Dessa maneira, uma motivação básica nada mais é do
que um agrupamento de técnicas de gamifição separadas e agrupadas
por um sentimento que motiva o usuário.

Estas motivações básicas estão desenhadas dentro de uma figura para
compor uma organização. Esta figura contém oito pontas, por isso o nome:
\textit{Octalysis}. A figura \ref{fig:octalysisframework} represernta esta
disposição no diagrama.

\begin{figure}[h]
    \centering
    \includegraphics[width=400px, scale=1]{figuras/octalysisframework}
    \caption{\textit{Octalysis} Framework}
    \label{fig:octalysisframework}
\end{figure}

Dessa forma, a seguir, sãoexplicadas as oito motivações básicas,
apontando quais são as suas respectivas técnicas, bem como
suas devidas explicações.

\subsubsection{Significado Épico e Chamado}
\label{sub:significadoepico}
Esta motivação básica está em torno de fazer com que o usuário sinta que
está fazendo algo épico e ajudando a sociedade, sendo altruísta com toda
a população. Fazendo-o acreditar que pode mudar algo muito importante
na vida de várias pessoas. Projetos \textit{Open} \textit{Source} são uma clara evidência
disto, onde o desenvolvedor trabalha e contribui para a comunidade, ajudando
todos.

Também entra nesta fase o fato de um jogador entrar no jogo e ganhar um
privilégio aparentemente muito valioso. Onde este acredita que teve uma
oportunidade que nenhum outro jogador, acreditando que ganhou a sorte de
principiante.

A seguir estão descritos as técnicas desta motivação básica:

\begin{enumerate}
    \item Narrativa: uma história que se inicia ao começar um determinado
        jogo, colocando o usuário dentro do enrendo e aplicando o contexto
        do "\textit{Modus Operants}";
    \item Sorte de Principiante: sorte daquele que acredita que recebeu
        um benefício único e nenhum dos outros jogadores recebeu algo
        parecido;
    \item Lanche Grátis: técnica que atribui ou presenteia um dado usuário
        com algo que tem uma grande dificuldade de ser alcançado ou é
        caro e bastante dispendioso para o usuário.
        O ponto é que logo após o lanche, o \textit{game} o incentiva a tomar
        algumas ações definidas;
    \item Elitismo: é a ação de aumentar e incentivar atitudes de orgulho
        de um grupo, fazendo com que os usuários destes sintam-se
        únicos e privilegiados, a fim de assegurar o orgulho do
        grupo inteiro, fazendo com que todos os integrantes sintam-se assim;
    \item Héroi da Humanidade: vem da técnica de que o jogador pode e deve
        ajudar a quem não consegue se ajudar, fazendo com que este tome
        atitudes e ações em prol de uma causa muito maior.
\end{enumerate}

\subsubsection{Motivação Desenvolvimento e Realização}
\label{sub:desenvolvimentoerealizacao}
A motivação básica de Desenvolvimento e Realização traz como base fazer
com que o jogador seja desafiado, que ele tenha metas a cumprir e que este
tem que desenvolver habilidades e fazer progressos para que o desafio
seja superado. É importante que nessa fase tenha recompensas e ganhos para
o jogador, como troféus, medalhas, emblemas, entre outros, para que o
usuário entenda que todo o esforço do desafio não foi em vão e que
teve um objetivo de recompensa por ele.

Exemplos desta motivação são os pontos, medalhas, \textit{rankings}, tabelas de
classificação, emblemas, entre outros. Que já são largamente utilizados
em várias plataformas atualmente.

Alguns exemplos de técnicas de gamificação para a motivação básica
de Desafio e Realização estão listadas a seguir:

\begin{enumerate}
    \item Pontos: um esquema de pontos aplicado para mostrar o progresso
        de um jogador em qualquer ponto desejado do objeto a ser gamificado;
    \item Símbolos de conquista e realização: que são emblemas que podem
        ser utilizados como espécie de reconhecimento sobre dada
        atividade ou desafio que o usuário desempenhou. Estas podem ser
        medalhas, emblemas, troféus, uniformes, estrelas, entre outros;
    \item Tabelas de classificação: são tabelas que vão mostrar como o
        jogador se porta diante dos demais jogadores, como
        estão estes resultados e o quanto é necessário para alcançar
        algum determinado objetivo. Esta técnica pode ser aplicada através
        de \textit{rankings} e tabelas.
    \item Barra de Progresso: permite que o usuário tenha uma clara visão
        e um bom \textit{feedback} do quanto ele está cumprido ou cumpriu dentro
        de um determinado objetivo.
    \item Escolhas Óbvias: são caminhos diferentes onde o usuário tem que tomar
        uma certa decisão ou fazer uma escolha. Neste momento, o usuário tem um
        escolha óbvia dentre todas as demais. No momento que ele escolher a óbvia
        irá se achar inteligente por ter conseguido identificar algo e ter
        feito a escolha correta.
    \item Oásis no Deserto: é uma recompensa que está sugerida e presente logo
        após determinadas escolhas que o usuário pode fazer;
    \item Efeito Estrela do \textit{Rock}: faz com que o usuário se sinta importante,
        com a ideia de que todos que estão na rede estão com vontade de interagir
        com ele, fazendo com que este se sinta importante, uma estrela do \textit{rock}.
\end{enumerate}

\subsubsection{Motivação Empoderamento e Feedback}
\label{sub:empoderamentoefeedback}
A motivação básica de empoderamento e \textit{feedback} é expressada quando os usuários
estão engajados em algum processo criativo e eles tem que tomar ações repentinas
e tentar diferentes combinações.

Para esta motivação básica o usuário não precisa apenas saber expressar
sua criatividade de várias maneiras, porém, além disso, precisa ser capaz
de ver os resultados das suas criações e seus respectivos feedbacks.

As técnicas que guiam esta motivação básica são as que estão listadas
a seguir:

\begin{enumerate}
    \item Escolhas significativas: esta técnica está em torno de fazer
        com que o usuário, mesmo com várias opções de montagem,
        tome ações corretas. O caminho correto não precisa ser exatamente o mesmo,
        pode ser um quebra-cabeças que você monta como desejar e no final
        consegue alcançar o objetivo esperado;
    \item Etapa desbloqueada: o usuário consegue ter acesso
        e utilizar novas funcionalidades, com novas possibilidades
        assim que uma etapa for concluída;
    \item \textit{Boosters}: Itens temporários que o usuário tem alguma
        capacidade aumentada, com mais poder, durante um período
        determinado de tempo;
    \item Feedback instantâneo: característica que permite com que o usuário
        tenha uma resposta imediata das ações que ele escolheu fazer e proceder;
    \item Controle de tempo real: trazendo e possibilitando que o jogador
        possa controlar ações e opções de um determinado objetivo deste
        em tempo real;
    \item Chain Combos: um conjunto de ações que traz recompensa para o
        usuário, porém, quando feitos em seguida, como um combo, tem
        efeitos e ganhos maiores.
\end{enumerate}

\subsubsection{Motivação de Propriedade e Posse}
\label{sub:propriedadeeposse}
É a motivação que gira em torno de mostrar ao usuário e fazer com que
este acredite que ele tem posse sobre algo que está interagindo na
plataforma. Quando este, por exemplo, passa tanto tempo utilizando
ou personalizando algo que passa a sentir que aquele dado objeto
é propriedade dele.

Um exemplo que pode ser aplicado é a utilização de dinheiro e riquezas
virtuais, como moedas e \textit{bitcoins}. De toda forma, um acumulo de riquezas
em geral contempla esta técnica.

\begin{enumerate}
    \item Construir do zero: faz com que o jogador sinta que ele está
        fazendo algo do início, e não que está recebendo algo que
        já está pronto para que este trabalhe;
    \item Coleção: transmite a ideia de um conjunto de itens que se todos
        estiverem juntos e reunidos, este irá se sentir completo.
    \item Pontos permutáveis: faz com que o usuário consiga utilizar
        seus pontos para adquirir algo que é caro por padrão;
    \item Monitor \textit{Attachment}: faz com que o jogador sinta e tenha a
        sensação que é dono de algo devido ao longo monitoramento
        da atividade que está desempenhando;
    \item Efeito Alfred: este é definido quando os usuários
        sentem que um produto ou serviço é tão personalizado às
        suas próprias necessidades que não é possível fazer isso
        de nenhuma outra forma;
    \item Avatar: quando um jogador consegue criar um perfil que pode ser
        personalizado e permite com que este esteja próximo aos gostos e
        intenções do usuário.
\end{enumerate}

\subsubsection{Influência Social e Pertencimento}
\label{sub:influenciasocialepertencimento}
Esta motivação básica utiliza de elementos e fatos sociais de forma
a incentivar as pessoas à pensamentos e ações em grupos e comunidades,
de forma social.

Essa motivação básica vai de encontro com a característica de ver algum
amigo, conhecido ou familiar, que possui uma dado atributo ou expertise
em dada atividade. Você, rapidamente, analisando que não possui a mesma
habilidade, ou não no mesmo nível, se esforça e engaja para que consiga
alcançá-la e estar próximo de tal.

As técnicas que regem e que estão presentes na Influência Social e Pertencimento
são as listadas a seguir:

\begin{enumerate}
    \item Mentoria: no momento em que uma pessoa que possui mais experiência
        em um dado assunto e orienta aqueles que estão começando nesta, pode
        ser enquadrado nesta técnica;
    \item Vangloriar-se: mostrar e se apresentar aos demais usuários como
        alguém que possui uma determinada qualidade que é bem reconhecida
        por todos que estão em sua volta;
    \item Prateleira de Troféus: uma gama de troféus, recompensas e conquistas
        que estão bem aparentes para os demais usuários olharem e perceberem
        através de qualquer meio que foi estabelecido;
    \item Desafio em Grupo: dasafios que podem ser elaborados com ou dentro
        de um determinado grupo, que incentiva as pessoas a trabalharem
        em uma ação conjunta e não em algo isolado, com a utilização de atividade
        individual;
    \item Tesouro Social: são pontos, presentes e benefícios que podem ser atribuídos
        para você a partir de um determinado jogador ou amigo que se encontra dentro
        do círculo de amigos;
    \item Orgulho Social:  são ações pequenas, de pouco esforço, que auxiliam e contribuem
        para o convívio social, como eventos pequenos e pouco significativos,
        como um \textit{like} em uma determinada rede social, ou um compartilhamento;
    \item Âncora de Conformidade: esta técnica possui base no que já temos ciência
        através de nossa cultura;
    \item \textit{Water} \textit{Cooler}: esta técnica consiste em disponibilizar algum local
        aberto e comum para que as pessoas consigam escrever, expor e falar
        sobre fatos e atividades aleatórias, o qual agrada um determinado grupo.

\end{enumerate}

\subsubsection{Escassez e Impaciência}
\label{sub:escassexeimpaciencia}
Esta motivação leva o jogador a sentir-se ansioso e receoso com a espera de algo
que ele ainda não tem e para que consiga, tem que esperar por algum tempo.

Um exemplo bem utilizado deste ponto é a utilização dos \textit{games} por meio de dinâmicas
agendadas, onde esses dizem ao usuário para retornar dentro de determinado tempo.
E caso o jogador não queira esperar por este tempo, deve utilizar algo que é caro
para este.

Um exemplo disto é a aplicação do Facebook quando iniciou, que tinha um início que não permitia
que todos utilizassem a plataforma, apenas se houvesse um convite por parte de um
determinado conhecido que já está dentro da base.

As técnicas que regem esta motivação básica são as seguir:

\begin{enumerate}
    \item \textit{Dangling}: esta técnica deixa bem clara para o jogador que ele não pode
        ter algo que é bem gratificante. Que para ter deve esperar ou adquirir
        de uma forma cara e dispendiosa;
    \item \textit{Anchored} \textit{Juxtaposition}: esta técnica concede duas opções para o usuário.
        Uma delas custa dinheiro e a outra custa e exige bastante tempo e esforço
        por parte do usuário;
    \item \textit{Torture} \textit{Breaks}: esta técnica faz com que o usuário seja obrigado a esperar
        de qualquer forma para obter algo. Neste caso, existe um ponto em que
        se o usuário ficar por um período longo de tempo esperando esta ação
        utilizando o jogo, a gamifição já estará sendo bem aplicada;
    \item \textit{Evolved} UI: fazer com que as pessoas tenham poucas opções no começo da
        trajetória, porém, com o seu desenvolvimento, estas opções vão aumentando.
\end{enumerate}

\subsubsection{Imprevisibilidade e Curiosidade}
\label{sub:imprevisibilidadeecuriosidade}
Esta motivação básica gira em torno de envolver o usuário com atitudes e ações que são
imprevisíveis, que o usuário não tem noção sobre o resultado que poderá receber.

Isto faz com que o usuário permaneça com a mente ocupada cogitando o que poderá
acontecer com o evento imprevisível.

As técnicas que guiam esta motivação básica são as listadas a seguir:

\begin{enumerate}
    \item Ovo de Páscoa: esta é uma notícia ou surpresa que agrada o usuário,
        que nasce de uma ação que não é esperada;
    \item \textit{Mystery} \textit{Boxes}: são recompensas que podem ser qualquer coisa
        logo após que alguma ação ou evento for concluído;
    \item \textit{Visual} \textit{Storytelling}: fazer com que o usuário tenha informações
        advindas de formatos de livros e histórias
        visuais;
    \item \textit{Oracle} \textit{Effect}: esta técnica está em torno de fazer com que o
        jogador pense sobre o que está por vir, se é algo positivo ou não;
    \item \textit{Russian} \textit{Roulette}: é a prática onde de tempos em tempos algum dado ponto
        ou jogador deve ser penalizado.
\end{enumerate}

\subsubsection{Perda e Rejeição}
\label{sub:perdaerejeicao}
Esta motivação básica trabalha e se baseia em construir algo para que algo ruim não
aconteça com o jogador, ou seja, na prevenção de acontecimentos ruins.

São práticas como evitar que o jogador perca todas as atividades que desempenhou
até então, ou descobrir que todo o progresso foi em vão.

Estas atitudes motivam o usuário a executar determinados feitos com base no que
ele não quer que aconteça.

As seguintes técnicas ilustram como esses sentimentos são aplicados no meio:

\begin{enumerate}
    \item \textit{Reghtful} \textit{Heritage}: esta técnica está firmada em fazer com que o usuário
        acredite que algo é dele e pertence a ele, porém, depois de algum tempo,
        se o usuário não desempenhar determinadas ações, este terá este ponto
        perdido;
    \item \textit{Evanescente} \textit{Opportunities}: algo que não vai aparecer jamais
        caso o usuário não tome uma ação requerida rapidamente;
    \item \textit{Countdown} \textit{Timers}: são esquemas de contagem regressiva para determindas
        atividades, que tem um tempo máximo para se concluir;
    \item \textit{Status} \textit{Wuo} \textit{Sloth}: tendência de mostrar ao jogador que este não irá evoluir ou melhorar se continuar
        a tomar atitudes como estão sendo tomadas;
    \item \textit{FOMO} \textit{Ponch}: este é um ponto muito forte da Motivação Básica 8, que tem como
        pretexto gerar no jogador um medo de perder, seja já qual for o que
        este adquiriu.
\end{enumerate}

\subsubsection{Lado Esquerdo e Direito do Cérebro}
\label{sub:leftright}
O \textit{Octalysis} \textit{framework} está dividido em duas partes, tais quais o \cite{chou2015actionable} nomeia
de \textit{Left Brain e Right Brain}, sendo que cada um destes é correspondente com qual lado do cérebro
as técnicas vão atuar. Cada motivação está posicionada em um dado local, que
representa um pensando de uma dada forma ou o contrário. Este posicionamento
da MB em cada lado representa qual lado do cérebro é utilizado.

A figura \ref{fig:octalysisleftright} ilustra esta divisão entre as técnicas.

\begin{figure}[h]
    \centering
    \includegraphics[width=400px, scale=1]{figuras/octalysisleftright}
    \caption{\textit{Left Brain and Right Brain}}
    \label{fig:octalysisleftright}
\end{figure}

Pode ser visto que existem técnicas do lado direito e do lado esquerdo. As
motivações básicas do lado direito estão mapeadas também com o lado direito
do cérebro, o qual tem ligação com as atividades lógicas, de raciocínio lógico.
Já as motivações básicas do lado esquerdo represetam ações mais criativas,
sentimentais e imprevisíveis.

\subsubsection{Sentimentos bons e ruins}
\label{sub:whiteblack}
As motivações básicas estão muito ligadas à característica de pensamento e
sentimento que é gerado no jogador. Cada motivação básica gera um tipo de
sentimento e vontade diferente no usuário.

Estas diferenças são representadas pela parte superior, representadas
pela \textit{White Hat}, que são sentimentos e motivações boas, que trazem
boas impressões para o jogador. A divisão é feita exatamente como
foi explicada na sessão \ref{sub:leftright}, porém, agora a
divisão é feita entre a parte superior ou inferior do \textit{framework}.

Porém, a parte inferior do \textit{framework} representa sentimentos ruins, que o jogador
vem a sentir no momento que participa e utiliza determinada MB. Já a
parte superior representa boas motivações, que ele chama de \textit{White Hat}.

Todas essas disposições podem ser vistas na figura
\ref{fig:octalysiswhiteblack} a seguir.

\begin{figure}[h]
    \centering
    \includegraphics[width=400px, scale=1]{figuras/octalysiswhiteblack}
    \caption{\textit{White Hat and Black Hat}}
    \label{fig:octalysiswhiteblack}
\end{figure}

\subsection{Fases da Gamificação}
\label{sub:fasesgamifição}
Todos os produtos que as pessoas utilizam na internet possuem diferentes
fases ao longo do seu ciclo de vida. Cada fase é reponsável por um tipo de contato diferente
do usuário com a interface e com a imersão em que este está submetido.

Cada fase representa um sentimento diferente, uma experiência diferente
e uma nova forma de se lidar com aqueles atributos referentes ao que está
em escopo no procedimento de interação com o produto propiciado.

Essas fases que cada projeto é submetido já são conhecidas e desenhadas. As fases
são quatro, bem claras e definidas. Elas são as seguintes:

\begin{enumerate}
    \item Descoberta;
    \item Reconhecimento;
    \item Construção;
    \item Fim de jogo.
\end{enumerate}

Essas fases circundam o ciclo de vida de um produto, desde o momento que este
é apresentado ao público até o momento que é deixado por ele.

A definição das fases é ilustrada claramente nos subcapítulos que virão a seguir.

\subsubsection{Descoberta}
\label{sub:descoperta}
É a fase onde o usuário não conhece sobre o produto, não tem noção de quais são
os
seus objetivos nem como pode utilizá-lo. Esta é a fase onde o usuário tem o primeiro
contato, onde percebe como este funciona, bem como seus conceitos e valores.

Um exemplo de descoberta é uma apresentação de uma página no facebook, onde,
o novo produto é demonstrado para grupos e nichos de interesse. A partir
de então, o usuário poderá passar a conhecer e utilizar o sistema.

Resumidamente, esta fase é reponsável por aprensentar o produto, fazer
com que os usuários o conheça.

\subsubsection{Reconhecimento}
\label{sub:reconhecimento}
Esta fase é reponsável por demonstrar ao usuário como o sistema se comporta.

Ela é essencial para que este entenda como o sistema funciona e o que cada
componente executa. Um exemplo bem conhecido desse procedimento é a utilização
de tutoriais e guias para novos usuários, no momento da sua chegada.

Ela termina quando o usuário está apto a continuar a utilizar o site sem
necessidade de aprender muitas outras novas ferramentas e funcionalidades.

Quando este está apto para tal, inicia-se a maior fase, onde o usuário
vai de fato entender e conhecer sobre o procedimento que está lidando.

\subsubsection{Construção}
\label{sub:constru_o}
Esta é a fase responsável pela real utilização do produto, onde as \textit{features}
de fato são utilizadas e irão agregar valor ao usuário.

Nesta parte o usuário já sabe e entende o papel de cada funcionalidade. Ele é capaz
de atingir os objetivos propostos. Aqui os recursos propostos são utilizados
a depender na experiência e conexão do usuário com o produto.

Aqui tem que ser criados gatilhos para que mantenha o usuário constantemente utilizando
o sistema de acordo com o planejado.

\subsubsection{Fim de Jogo}
\label{sub:fim_de_jogo}
Toda aplicação desenvolvida passa pela fase de partida, onde é totalmente utilizada
e de alguma forma, o usuário a deixará.

Não necessariamente deixará de utilizar e participar do envolvimento total proposto pela
organização. Um exemplo disto é um jogo desenvolvido. Quando o primeiro jogo acabar, o
usuário passará pela fase de fim de jogo, que pode deixar o usuário motivado a se conectar
e adquirir a próxima versão do jogo que é lançada futuramente.

É importante que seja feito corretamente o desfeixo do produto para que uma linhagem seja
prosseguida.


Todas essas diretivas e fases que existem dentro do ciclo de vida de um produto
devem ser
tratadas de forma independente e diferente entre si. Agora pode-se indagar onde
a gamificação
entra neste processo, sendo que cada fase deve ser tratada de uma forma diferente pelo
usuário e, consecutivamente, por parte de quem está a oferecer o produto.

Assim, há a necessidade de que a gamificação também seja moldada conforme o objetivo de
cada fase a ser aplicada.

Dessa forma, cada fase implementada é pensada e avaliada para que seja possível
aplicação de um  projeto de gamificação. Cada fase terá um foco em motivações
básicas diferentes, que propiciarão uma experiência diferente para o usuário.

A figura \ref{fig:fasesoctalysis} ilustra um exemplo do como pode ser aplicado na
Rede Social About a gamificação ao longo das quatro fases.

\begin{figure}[h]
    \centering
    \includegraphics[width=400px, scale=1]{figuras/fasesoctalysis}
    \caption{Fases do \textit{Octalysis}}
    \label{fig:fasesoctalysis}
\end{figure}

Como pode ser visto na figura \ref{fig:fasesoctalysis}, são projetados vários
desenhos e \textit{designs} modificados e diferentes para cada fase. Cada uma destas
tem um pensamento e objetivo diferente.

Na fase de descoberta, pode ser visto que a motivação básica mais presente é
a imprevisão e a curiosidade. O que dá margem para que o usuário imagine diferentes
possibilidades sobre o produto.

No momento de uma propaganda, por exemplo, este lado do \textit{framework} pode gerar uma
extrema curiosidade no usuário, o que fará com que ele fique motivado a procurar
e entender mais sobre o que está sendo anunciado.

Isto pode ser extremamente importante para conseguir capturar novos usuários.

Na segunda fase, em que o usuário vai conhecer sobre o produto, pode ser visto
que as fases relativas a desenvolvimento próprio e realização de si mesmo
são bem mais presentes.

Este ponto pode ser aplicado, pois o usuário irá se sentir realizado e inteligente
ao observar seu desenvolvimento próprio elevado. Isto irá gerar um prazer em fazê-lo
sentir o quanto pode ser bom em realizar as tarefas que a ele estão sendo designadas
no início do procedimento.

Na terceira fase é possível verificar que duas motivações básicas são muito presentes:

\begin{itemize}
    \item Motivação Básica Cinco: Influência e Dinâmica Social;
    \item Motivação Básica Seis: Escassez e Impaciência.
\end{itemize}


Para a Motivação Básica Cinco, isto deixa o usuário motivado ao utilizar o produto
por sentir que está exercendo uma alta influência social, que está envolvido em
uma dinâmica social que faz influência em outras pessoas.

Isto faz com que o usuário fique motivado a continuar engajado no processo, pois
este estará conseguindo perceber o quanto está sendo participativo no meio social
e que o produto está sendo proveitoso por fazê-lo se sentir socialmente influente
e participativo.


A segunda motivação básica vizualizada nesta fase, Escassez e Impaciência, acontece
pois é possível verificar que o usuário fique motivado a executar determinadas
tarefas baseado neste sentimento.

Esta o deixará preocupado com a questão de não cumprir corretamente os objetivos.
Esta fase é responsável por fazê-lo se sentir em um meio escasso caso não execute
os objetivos propostos.

Isto vai motivar o usuário e vai fazer com que faça o necessário para que não
sinta estes sentimentos.

A última fase, fim de jogo, também tem sua motivação básica predominante que
a guia. Esta é guiada pela Motivação Básica Oito: perca e rejeição.

Esta irá gerar um sentimento que faz o usuário se sentir mal. Este sentimento
envolve o fato de que o usuário pode perder todo o processo que foi executado.



Este é um sentimento ruim. Sentimento qual o usuário não deseja sentir. Para tanto
ele se esforçará a fim de não presenciar as experiências que são submetidas.

Como pode ser visto, estes procedimentos de cada fase são extremamente aplicáveis
e úteis para que o usuário tenha várias experiências ao longo do clico de vida do
produto. O que propiciará uma experiência muito mais agradável.

Dessa forma, são desenhados quatro frameworks diferentes para a Rede Social About.
Uma para cada fase diferente do produto, onde são estudadas separadamente para
aplicá-las e possibilitar uma boa experiência para o usuário.

\subsection{\textit{Octalysis Strategy Dashboard}}
\label{sec:octalysisdashborad}
O \textit{framework} \textit{Octalysis} oferece suporte para a construção de um projeto de gamificação
bem estruturado e baseado em necessidades do domínio do problema.

Este suporte se trata do \textit{Octalysis} \textit{Strategy} \textit{Dashboard}, o qual pode ser analisado
 as
estretégias de mercado, perspectiva do usuário, intenções desejadas para a gamificação,
mecanismos de \textit{feedback} e incentivos.

Existem processos sistematizados para estabelecer cada fase e como é dado o
resultado da gamificação.

Para este trabalho, são utilizados estes procedimentos sistematizados.

Para ilustrar a metodologia de estratégia do \textit{Octalysis} \textit{dashboard}, é representada
a figura a seguir, que contém a metodologia e a formalização da sua construção.

A seguir são descritos subcapítulos, que retratarão o papel e a utilidade de
cada
componente.


 \begin{figure}[h]
     \centering

     \includegraphics[width=450px, scale=1]{figuras/dashboard}
     \caption{\textit{Octalysis} \textit{Strategy} \textit{Dashboard}}

     \label{fig:dashboard}
 \end{figure}

\subsubsection{Métricas de Negócio}
\label{sub:business_metrics}
As métricas de negócio, são termos quantitativos que podem ser utilizados
para ter um número palpável sobre como está um determinado ponto do projeto de gamificação
que teve como o objetivo de ser atacado.

Essas métricas, irão auxiliar a verificarmos o quanto a aplicação da gamificação
 foi eficaz ou
não dentro de um determinado objetivo.

Alguns exemplos de técnica de gamificação que são utilizadas estão a seguir:

\begin{itemize}
    \item Aumentar o número de seguidores dos usuários prêmio;
    \item Aumentar o número de vendas de um livro sobre o produto;
    \item Aumentar o número de inscritos na rede social;
    \item Aumentar a quantidade de acessos diários;
    \item Aumentar os seguidores inscritos;
    \item Aumentar os usuários que compartilham conteúdos pelas redes sociais;
    \item Aumentar a quantidade de curtidas em determinado post.
\end{itemize}

Estes exemplos de métricas são submetidos à Rede Social About antes da apresentação da
gamificação. E assim que determinada técnica for utilizada, é então executada
 uma
segunda medição, que propiciará analisar as diferenças entre os resultados obtidos.

\subsubsection{Definir Tipos de Usuário}
\label{sub:define_user_types}
Este ponto do \textit{dashboard}, para definir os tipos dos usuários, é responsável por conseguir
elaborar e definir quais são os tipos de usuários que são almejados e trabalhados, quando
falamos sobre gamificação.

Esta fase é um processo de definição de nicho sobre onde a gamificação vai atuar, quanto a
usuários, dentro da Rede Social About? Quais são os passos utilizados para que este público
seja atingindo?

Alguns exemplos de tipos de usuário se encontram a seguir:

\begin{itemize}
    \item Companhias que desejam que seus trabalhadores atinjam determinadas métricas
        ao fim de cada mês;
    \item Educadores e políticos que querem utilizar conhecimento para criar impactos
        sociais;
    \item Indivíduos que são apaixonados por gamificação, \textit{games} e desenvolvimento próprio.
\end{itemize}

Desta maneira, é possível realizar um projeto de gamificação focado ao definir o tipo
de usuários. Pois, a partir daí, é possível identificar quais caminhos são mais vantajasos
quanto a escolha das motivações básicas que são utilizadas ao longo das quatro fases.

\subsubsection{Definir Ações Desejadas}
\label{sub:define_desired_actions}
A definição das ações desejadas são todas as iniciativas tomadas pelo usuário que o levam a caminhar para
o \textit{Win} \textit{Stade} (Estado de Vitória), seja ela em qual fase for. Sendo assim, a Rede Social
About terá alguns pontos que são definidos como os desejados. Estes são desenhados
até que o estado de vitória seja definido. Assim, para as quatro fases são definidas
ações diferentes. Alguns exemplos de ações que podem ser escolhidas são apresentadas
a seguir.

Ações na fase da descoberta:
\begin{itemize}
    \item Conhecer a Rede Social About;
    \item Clicar no link da Rede Social About;
    \item Conhecer as \textit{features} oferecidas pela Rede Social.
\end{itemize}


Ações na fase de reconhecimento do projeto:
\begin{itemize}
    \item Executar o tutorial de uso da About;;
    \item Compartilhar a Rede Social About com os amigos;
    \item Adicionar foto e email na \textit{network};
    \item Permitir a inscrição na lista de email.
\end{itemize}

Já para a fase de construção do projeto, os seguintes pontos podem ser um
exemplo:

\begin{itemize}
    \item Fazer login diariamente na \textit{network};
    \item Abrir semanalmente os emails enviados pela \textit{network};
    \item Compartilhar abouts com os amigos;
    \item Participar de grupos no facebook sobre a rede social about;
    \item Adquirir a versão prêmio da rede social about;
    \item Inscrever em grupos de discussão sobre a rede social about;
    \item Escrever mais de um about diariamente;
    \item Votar em mais de vinte abouts diários.
\end{itemize}

Por fim, na fase de fim de jogo, alguns exemplos de construção podem ser dados.
Eles são os seguintes:
\begin{itemize}
    \item Se tornar contribuidor da Rede Social About;;
    \item Fazer parte da equipe de desenvolvedores da About;
    \item Propor melhorias para a about;
    \item Tornar-se moderador dos abouts.
\end{itemize}

Estes exemplos ajudam e esclarecer como os objetivos podem ser alcançados. Elas
definem um nível de
granularidade maior.

\subsubsection{Definir Mecanismos de \textit{Feedback}}
\label{sub:define_feedback_mechanics}
A definição de mecanismos de \textit{feedback} são extremamente importantes para a experiência do usuário
com a \textit{network}. Este é responsável por ilustrar e deixar bem claro para o usuário
, como ele está
prosseguindo no desenvolvimento do projeto.

Atualmente os usuários tem requirido feedbacks constantes, em tempo real, para as suas ações
realizadas. Sendo assim, é necessário que existam esses gatilhos em vários pontos da
Rede Social About e que o usuário possa entender rapidamente.

A seguir estão alguns exemplos de como podem ser esclarecidos esses feedbacks para o usuário:

\begin{itemize}
    \item \textit{Countdown} \textit{Timers};
    \item Desbloquear conteúdo da página;
    \item Status de progresso na \textit{sidebar};
    \item Verificação de qual era a melhor escolha;
    \item Vídeo embutido;
    \item Barra de pontos de status;
    \item Certificados;
    \item Medalhas;
    \item Gráficos de desempenho.
\end{itemize}

Assim, com exemplos dessa maneira, é possível que o usuário verifique o quanto suas atividades estão
sendo aproveitadas.

\subsubsection{Incentivos e Recompensas}
\label{sub:incentives_and_rewards}
O sistema de incentivos e recompensas fecham o ciclo do \textit{dashboard}, que fazem com que
os usuários se sintam motivados a alcançar cada estado de vitória. Eles ajudam a
indicar
o quanto ainda falta para que o estado seja almejado.

\begin{itemize}
    \item \textit{Status} \textit{Points}
    \item Símbolos de vitórias;
    \item Conhecer os desenvolvedores da about;
    \item Ter acesso a arquivos confidenciais;
    \item Descontos nos produtos.
\end{itemize}

\subsubsection{Objetos de Gamificação}
\label{sec:objetodegamificacao}
Os objetos de gamificação são os pontos da rede social em que é aplicado o \textit{framework},
com os objetivos de atingir alguma meta de negócio.

Os objetivos de gamificação são os seguintes:

\begin{itemize}
    \item Fazer com que o usuário escreva mais abouts;
    \item Fazer com que o usuário julgue mais abouts;
    \item Fazer com que o usuário convide amigos que não estão cadastrados na about.
\end{itemize}

