\begin{resumo}
    Nos últimos anos as redes sociais vem tendo grande visibilidade no
    mundo da tecnologia e estão se tornando um meio de comunicação e
    compartilhamento de informações muito utilizado pelos usuários do
    meio digital, como indicado por \cite{socialnetworkdefinition}. A
    gamificação, por sua vez, possui técnicas que motivam e engajam
    os usuários a executarem determinada atividade, como defendido por
    \cite{deterding2011gamification}. Este trabalho se trata de um produto
    de desenvolvimento tecnológico, que irá aplicar um \textit{Framework}
    de gamificação na Rede Social About(RSA). Após a aplicação, será executada
    uma coleta de dados para verificar se os objetivos propostos na gamificação
    foram verificados na RSA.

    %  O resumo deve ressaltar o objetivo, o método, os resultados e as conclusões 
    %  do documento. A ordem e a extensão
    %  destes itens dependem do tipo de resumo (informativo ou indicativo) e do
    %  tratamento que cada item recebe no documento original. O resumo deve ser
    %  precedido da referência do documento, com exceção do resumo inserido no
    %  próprio documento. (\ldots) As palavras-chave devem figurar logo abaixo do
    %  resumo, antecedidas da expressão Palavras-chave:, separadas entre si por
    %  ponto e finalizadas também por ponto. O texto pode conter no mínimo 150 e 
    %  no máximo 500 palavras, é aconselhável que sejam utilizadas 200 palavras. 
    %  E não se separa o texto do resumo em parágrafos.

    \vspace{\onelineskip}

    \noindent
    \textbf{Palavras-chaves}: gamificação. redes sociais. octalysis.
\end{resumo}
