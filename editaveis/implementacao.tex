\chapter[Estruturação e Implementação da Gamificação da About]{Estruturação e Implementação da Gamificação da About}

Para a implementação do trabalho, os seguintes
passos foram executados, cada um é
descrito em uma subsessão: execução do teste piloto, \textit{survey} para identificação das técnicas de gamificação,
análise estatística das técnicas, construção do \textit{framework} de gamificação, escolha do objeto de gamificação,
implementação das técnicas no objeto de gamificação.

Aqui será relatado todos os passos que foram tomados para a realização do trabalho.


% \section{Planejamento}
% \label{sec:planejamento}

% \section{Definição da Gamificação}
% \label{sec:definicao_gamification}

\section{Execução do Piloto}
\label{sec:execucao_do_piloto}
Com a intenção de validar a aplicação da About, foi implementado um piloto com as suas funcionalidades
básicas. Aqui existia apenas um post, onde o administrador um da aplicação escolhe dentre temas propostos
pelos usuários, que era postado a cada três dias.
Nesta sessão iremos descrever sobre como esta implementação foi executada.

Para desenvolver e aplicar este piloto, será necessário executar um procedimento
com os seguintes passos:

\begin{enumerate}
    \item Definir tecnologia;
    \item Desenvolver solução;
    \item Implantação em produção;
    \item Aplicar \textit{marketing} da solução para um público reduzido;
    \item Manter solução;
    \item Finalizar solução.
\end{enumerate}


Primeiramente, haviam alguns pontos necessários ppara a implantação da aplicação. Assim como agilidade para colocar
em produção e desenvolvimento rápido. Estes foram os padrões seguidos para a produção do piloto:


\begin{itemize}
    \item Desenvolvimento rápido;
    \item Padrões de Design simples;
    \item Escalabilidade baixa;
    \item Os usuários não carecem de executar cadastros e logins;
    \item Suporte para questionários;
    \item Facilidade de implantação.
\end{itemize}

Assim, estes pontos foram avaliados. Desta forma, era necessário implementar um software
de alta produtividade, que entregasse a funcionalidade muito rapidamente. Assim, dados estes
pontos, a tecnologia escolhida foi o WordPress, utilizando a linguagem de programação PHP.

Esta possui vários módulos prontos, desenvolvidos pela comunidade e forneceidos de maneira
gratuita. Nestes módulos, temos componentes para questionários, design de interface já
implementados facilmente e de extrema facilidade para implantação.

Como o software é considerado pequeno, não houve nenhuma necessidade de preocupação com a
infraestrutura  do sistema e com fatores técnicos ligados à alta performace. Também não foi
necessário fazer um controle de acessos e de usuários, pois a votação seria baseada no IP
do cliente, não permitindo novos acessos daquele mesmo IP.

Escolhida a tecnologia, foi iniciada a implementação da aplicação. Esta consistiu em escolher o
módulo de enquete e o módulo de layout. Após isto, tudo estava finalizado.

Para enquetes, foi
utilizado o pacote WP-Polls, este pode ser encontrado em: \url{https://wordpress.org/plugins/wp-polls/}.

Para a implementação do layout foi utilizado o módulo Amadeus, que pode ser encontrado em:
\url{https://br.wordpress.org/themes/amadeus/}.

No âmbito da implantação foi necessário apenas comprar um servidor cloud e instalar os pacotes do WordPress
dentro deste. O código fonte da aplicação pode ser encontrado neste link:
\url{https://wordpress.org/download/}.

Implementada a solução e operando, foi necessário divulgá-la para aqueles que utilizariam o código.
Então foi criado um perfil na rede social FaceBook, onde este divulgou para todos os alunos de engenharia
de software o propósito do site, bem como cada um poderia utilizá-lo. Assim, foram feitas postagens diárias
na página falando sobre as notícias da aplicação. A comunidade em si se empolgou bastante com os novos resultados.

O perfil ganhou vários seguidores que participavam ativamente ou passivamente da plataforma, respondendo, criando e
visualizando os posts realizados no projeto piloto.

Para manter a solução, era necessário levantar enquetes para o público. Então, para manter o público interessado,
foi criado uma segunda enquete que sempre estava disponível perguntando aos usuários:

Qual seria o tema que eles desejavam ver no próximo post do piloto?

Assim, todos os temas mais pedidos pelo público era lançado e operando ao longo de dois dias para uma votação.

Por fim, foi declarado que seria finalizada a solução e que o último post seria feito. Este foi realizado para a
última votação. E assim o piloto saiu de produção.

\section{Levantamento das Técnicas de Gamificação}
\label{sec:gamifição}
Com base em uma pesquisa executada com os participantes do piloto, foi executado um levantamento das
técnicas básicas, para colocá-las frente aos objetivos estabelecidos no trabalho.

Primeiramente foi perguntado a alguns usuários que utilizaram a rede social para executar alguns
pontos relativos à impressão destes ao piloto. Detalhes sobre como estes viam as impressões passadas.
Claro que o intuito de todas as perguntas era idetificar como os usuários visualizavam dinâmicas
que envolviam gamificação dentro do piloto.

As perguntas realizadas aos usuários foram as seguintes:

\begin{quotation}
    Questão 01: Na sua opinião, o que este \textit{site} representava?
\end{quotation}

\begin{quotation}
    Questão 02: O que você acredita que motivava e levava as pessoas a utilizarem o \textit{site}?
\end{quotation}

\begin{quotation}
    Questão 03: E quanto ao contrário, o que você acredita que levava as pessoas a se desmotivarem e a não
    utilizarem mais o \textit{site}?
\end{quotation}

Assim, para estas perguntas, tivemos um total de 4 pessoas respondendo este questionário. A resposta de cada
usuário será descrita a seguir. Iremos dividir as etapas em questões. Listando assim as quatro respostas
para cada questão.


\subsection{Questão 01}
\subsubsection{Usuário 01}
A curiosidade levava as pessoas a entrarem na plataforma e descobrirem o que as
pessoas estavam falando das outras que você conhece.
\subsubsection{Usuário 02}
Era uma oportunidade de trazer a tona todos os sentimentos e sensações das pessoas
em relação as outras. Tanto relação a atitudes quanto a impressões sobre eventos
passados. Uma oportunidade de trazer visibilidade para a consequencia de uma ação
feita por aguém.
\subsubsection{Usuário 03}
Criativo
\subsubsection{Usuário 04}
Um site que surgiu na faculdade, que a princípio era um portal para relatado de
acontecimentos da faculdade UnB-FGA, onde existiam vários posts inimagináveis
de algumas pessoas. E era confirmado este relato quando você via os outros comentários
anônimos, reforçando exatamente aquele ponto. Eram comentários que a princípio ninguém
acreditaria, pois eram extremamente fortes. Chegando a chocar com os fatos
declarados.


\subsection{Questão 02}
\subsubsection{Usuário 01}
As pessoas estarem curiosas para saberem sobre segredos sobre outras, que não
são contados no cotidiano.
\subsubsection{Usuário 02}
oportunidade de Conseguir punir ações errônias para as pessoas por atitudes erradas.
Mostrando que atitudes erradas tem consequencia, expondo o autor caso ele faça
algo de ruim para aquele meio.
\subsubsection{Usuário 03}
curiosidade
\subsubsection{Usuário 04}
Você não queria ser um alvo do post da plataforma, você quer ser aquele que julga.
Então criava-se a curiosidade sobre os comentários feitos à minha pessoa e também
os comentários feitos a outras pessoas. O ambiente de 'fofoca' estimulava o ambiente.

\subsection{Questão 03}
\subsubsection{Usuário 01}
Medo de exposição de seus detalhes pessoas para demais individuos.
\subsubsection{Usuário 02}
A probabilidade de perder o controle diante da invasão de privacidade. E trazer
consequencias reais para a vida da pessoa.
\subsubsection{Usuário 03}
Falta de atualização e novidades em novos conteúdos no portal.
\subsubsection{Usuário 04}
Medo de que seu nome estivesse ali no meio da plataforma e a falta de existência
de um meio de defesa para os acusados.


\section{Survey das Técnicas}
\label{sec:gamifição}
Para apurar as técnicas capturadas no levantamento, foi executado um survey, a fim de dados para
embasar uma análise estatística.

\section{Análise Estatística}
\label{sec:gamifição}
Com base nestes dados, foram definidos quais as técnicas tem relação entre si para auxiliar no processo
da escolha destas para o Framework.

\section{Construção do Framework}
\label{sec:gamifição}
Definidas as técnicas para a implementação, estas foram utilizadas para a construção do Framework
de Gamificação.

\section{Objeto de Gamificação}
\label{sec:gamifição}
Assim que todo o framework foi montado, possibilitou então que fossem escolhidos em quais
pontos da About seria possível aplicá-los. Esta sessão irá discutir sobre todos os pontos de escolha
dos objetos de Gamificação.

\section{Implementação das Técnicas}
\label{sec:gamifição}
Assim que todo o framework foi estabelecido e seus objetos também, foi possível implementar
na rede social todos os pontos para colocar estas técnicas em prática.
