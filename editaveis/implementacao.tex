\chapter[Estruturação e Implementação da Gamificação da About]{Estruturação e Implementação da Gamificação da About}

Para a implementação do trabalho, os seguintes
passos foram executados, cada um é
descrito em uma subsessão: execução do teste piloto, \textit{survey} para identificação das técnicas de gamificação,
análise estatística das técnicas, construção do \textit{framework} de gamificação, escolha do objeto de gamificação,
implementação das técnicas no objeto de gamificação.

Aqui será relatado todos os passos que foram tomados para a realização do trabalho.


% \section{Planejamento}
% \label{sec:planejamento}

% \section{Definição da Gamificação}
% \label{sec:definicao_gamification}

\section{Execução do Piloto}
\label{sec:execucao_do_piloto}
Com a intenção de validar a aplicação da About, foi implementado um piloto com as suas funcionalidades
básicas. Nesta sessão iremos descrever sobre como esta implementação foi executada.

Para desenvolver e aplicar este piloto, será necessário executar um procedimento
com os seguintes passos:

\begin{itemize}
    \item Definir tecnologia;
    \item Desenvolver solução;
    \item Implantação em rodução;
    \item Aplicar \textit{marketing} da solução para um público reduzido;
    \item Manter solução;
    \item Finalizar solução.
\end{itemize}

Primeiramente, haviam alguns pontos necessários ppara a implantação da aplicação. Assim como agilidade para colocar
em produção e desenvolvimento rápido. Estes foram os padrões seguidos para a produção do piloto:


\begin{itemize}
    \item Desenvolvimento rápido;
    \item Padrões de Design simples;
    \item Escalabilidade baixa;
    \item Os usuários não carecem de executar cadastros e logins;
    \item Suporte para questionários;
    \item Facilidade de implantação.
\end{itemize}

Assim, estes pontos foram avaliados. Desta forma, era necessário implementar um software
de alta produtividade, que entregasse a funcionalidade muito rapidamente. Assim, dados estes
pontos, a tecnologia escolhida foi o WordPress, utilizando a linguagem de programação PHP.

Esta possui vários módulos prontos, desenvolvidos pela comunidade e forneceidos de maneira
gratuita. Nestes módulos, temos componentes para questionários, design de interface já
implementados facilmente e de extrema facilidade para implantação.

Como o software é considerado pequeno, não houve nenhuma necessidade de preocupação com a 
infraestrutura  do sistema e com fatores técnicos ligados à alta performace. Também não foi
necessário fazer um controle de acessos e de usuários, pois a votação seria baseada no IP
do cliente, não permitindo novos acessos daquele mesmo IP.

\section{Levantamento das Técnicas de Gamificação}
\label{sec:gamifição}
Com base em uma pesquisa executada com os participantes do piloto, foi executado um levantamento das
técnicas básicas, para colocá-las frente aos objetivos estabelecidos no trabalho.

\section{Survey das Técnicas}
\label{sec:gamifição}
Para apurar as técnicas capturadas no levantamento, foi executado um survey, a fim de dados para
embasar uma análise estatística.

\section{Análise Estatística}
\label{sec:gamifição}
Com base nestes dados, foram definidos quais as técnicas tem relação entre si para auxiliar no processo
da escolha destas para o Framework.

\section{Construção do Framework}
\label{sec:gamifição}
Definidas as técnicas para a implementação, estas foram utilizadas para a construção do Framework
de Gamificação.

\section{Objeto de Gamificação}
\label{sec:gamifição}
Assim que todo o framework foi montado, possibilitou então que fossem escolhidos em quais
pontos da About seria possível aplicá-los. Esta sessão irá discutir sobre todos os pontos de escolha
dos objetos de Gamificação.

\section{Implementação das Técnicas}
\label{sec:gamifição}
Assim que todo o framework foi estabelecido e seus objetos também, foi possível implementar
na rede social todos os pontos para colocar estas técnicas em prática.

