\chapter[Introdução]{Introdução}
As redes sociais tem se tornado extremamente populares na última década.
Várias
Redes Sociais despontaram e tomaram proporções grandes, tendo uma quantidade enorme de usuários
participando destas. Alguns exemplos são: \textit{MySpace}, \textit{Facebook}, \textit{Twitter},
\textit{ByWorld}, entre outras, que conseguiram milhões de usuários, onde a maioria integra
as suas funcionalidades com hábitos diários praticados pelos usuários.
Estas redes sociais estão repletas de tecnologias e funcionalidades diferentes,
trazendo características e suportando uma vasta quantidade de interesses e práticas
entre as pessoas. Na maioria das vezes, apoiado na persuasão e na presença social
dos indivíduos.

A Rede Social About (RSA) tem o propósito de dar transparência às personalidades de seus usuários, permitindo com que todos
estes saibam sobre algum aspecto de qualquer outro usuário, desde que ambos tenham aceitado os
termos de consentimento preestabelecidos.

Já a Gamificação, para a definição de \cite{chou2015actionable},  é o ato de cuidadosamente aplicar ao mundo
real e as atividades produtivas os elementos divertidos e envolventes dos jogos.
É a ação de  motivar e engajar os usuários a executarem alguma determinada
tarefa. 

Neste trabalho, será elaborado, definido e aplicado um \textit{Framework} de Gamificação na RSA 
para motivar e engajar os seus usuários a utilizarem-na, aplicando uma abordagem proposta por \cite{chou2015actionable}.

\section{Problema}
A mecânica e característica básica da RSA já contém gamificação na sua essência, se tratando dos comentários e votações anônimas.
Estas técnicas já existentes auxiliam a motivar o usuário na RSA na fase de
entrada e descoberta. Porém, não são suficientes para o manter na fase de
construção, que é a maior parte da vida útil da aplicação. Desta forma, há
a necessidade de manter o usuário motivado e engajado para as outras partes
do ciclo de vida.

\section{Objetivos}
Foram separados os objetivos do trabalho entre gerais e específicos. Estes, serão
descritos a seguir.
\subsection{Objetivos Gerais}
Aplicar um \textit{Framework} de  gamificação adaptado para a Rede Social About,
com a finalidade de motivar e engajar os usuários a executarem determinadas
ações, que propiciará melhores envolvimentos entre os usuários.
\subsection{Objetivos Específicos}
\begin{itemize}
    \item Definição do \textit{Framework} de Gamificação para a Rede Social About,
        o qual será adaptado para as necessidades de negócio deste, possibilitando
        motivação e engajamento para os usuários;
    \item Implementação da Gamificação na RSA, aplicando, programaticamente, as técnicas
        de gamificação definidas no \textit{Framework}, disponibilizando quadros, botões,
        tabelas, \textit{ranking} e os demais pontos definidos;
    % \item Coleta dos resultados da aplicação da Gamificação, analisando o comportamento da
    %     RSA após a aplicação das técnicas de gamificação, possibilitando que seja verificado
    %     se a aplicação obteve a sua efetividade.
\end{itemize}
\section{Motivação}
A gamificação tem por fim o objetivo de 
possibilitar, no final das contas, engajamento e motivação aos usuários para
executarem determinada tarefa. Isso faz com que os usuários estejam mais
motivados para usar a Rede Social About. 

A questão da motivação é de extrema importância para as redes sociais, mostrando
que assim é possível fazer com que os usuários sintam prazer e estejam
contentes ao utilizar a plataforma. Dessa forma, com essas intenções,
será aplicado o \textit{Framework} de Gamificação.

\section{Metodologia}
O resultado deste trabalho é tido como um produto de desenvolvimento tecnológico, no qual será implementado
um entregável funcional de tecnologia apoiando a motivação e engajamento dos usuários.

\subsection{Classificação da Pesquisa}
\label{sub:classifica_o_da_pesquisa}
Este trabalho tem a classificação, de acordo com \cite{gil2010metodos}, como desenvolvimento tecnológico,
onde, a partir de algumas necessidades e características, será desenvolvido um produto
de tecnologia para a sociedade.

\subsection{Referencial Teórico}
\label{sub:referencial_te_rico}
O referencial teórico foi elaborado utilizando pesquisas em livros digitais, artigos e
revistas indexados em três bases bibliográficas: \textit{Scopus}, \textit{IEEE}, \textit{Science Direct}. Todos estes
foram pesquisados através dos acessos disponibilizados pela Universidade de Brasília.
Alguns outros artigos e revistas foram escolhidos utilizando a técnica de
\textit{Snowball} para trás em no máximo, dois níveis.


\section{Estrutura de Monografia}
Este trabalho foi dividido em quatro capítulos principais. O capítulo 1 apresenta a introdução
do trabalho. O capítulo 2 trata sobre o que se tem de consolidado na academia em volta
dos temas que serão tratados. No capítulo 3 será apresentada a proposta do trabalho, que conterá
os desafios que este propõe tratar. O capítulo 4 tem o viés de contextualizar como será
desenvolvida e definida a proposta do trabalho. Por fim, temos o capítulo 5 com o resultado do
trabalho, contemplando a implementação da gamificação na RSA.
