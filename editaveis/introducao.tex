\chapter[Introdução]{Introdução}
As redes sociais tem se tornado extremamente populares na última década.
Várias
Redes Sociais despontaram e tomaram proporções grandes, tendo uma gama grande
de usuários participando destas. Alguns exemplos são: \textit{MySpace}, \textit{Facebook}, \textit{Twitter},
\textit{ByWorld}, entre outras, que conseguiram milhões de usuários, onde a maioria integra
as suas funcionalidades com hábitos diários praticados pelos usuários.
Estas redes sociais estão repletas de tecnologias e funcionalidades diferentes,
trazendo características e suportando uma vasta quantidade de interesses e práticas
entre as pessoas. Na maioria das vezes, apoiado na persuasão e na presença social
dos indivíduos.

A Rede Social About (RSA) tem o propósito de dar transparência às personalidades de seus usuários, permitindo com que todos
estes saibam sobre qualquer aspecto sobre qualquer outro usuário, desde que ambos tenham aceitado os
termos de consentimento pré estabelecidos.

Já a Gamificação, para a definição de \cite{chou2015actionable},  é o ato de cuidadosamente aplicar ao mundo
real e as atividades produtivas os elementos divertidos e envolventes dos jogos.
É a ação de enganjar e motivar os usuários a executarem alguma determinada
tarefa. 

Neste trabalho, será elaborado, definido e aplicado um \textit{Framework} de Gamificação na RSA 
para enganjar e motivar os seus usuários, utilizando uma abordagem proposta por \cite{chou2015actionable}.


\section{Objetivos}
Foram separados os objetivos do trabalho entre gerais e específicos. Estes, serão
descritos a seguir.
\subsection{Objetivos Gerais}
Aplicar um \textit{Framework} de  gamificação adaptado para a Rede Social About.
\subsection{Objetivos Específicos}
\begin{itemize}
    \item Definição do \textit{Framework} de Gamificação;
    \item Implementação da Gamificação na RSA;
    \item Coleta dos resultados da aplicação da Gamificação.
\end{itemize}
\section{Problema}
Nos vários cenários de redes sociais existentes, é muito comum o lançamento de cases
que não chamam a atenção dos usuários, as quais não motivam estes a se sentirem
motivados e enganjados.
\section{Motivação}
A gamificação tem por fim o objetivo de 
possibilitar, no final das contas, engajamento e motivação aos usuários para
executarem determinada tarefa. Isso faz com que os usuários estejam mais
motivados para usar a rede social about. 

A questão da motivação é de extrema importância para as redes sociais, mostrando
que assim é possível fazer com que os usuários sintam prazer e estejam
contentes ao utilizar a plataforma. Dessa forma, com essas intenções,
será aplicado o \textit{Framework} de Gamificação \textit{Octalysis}.

\section{Metodologia}
\subsection{Classificação da Pesquisa}
\label{sub:classifica_o_da_pesquisa}
Este trabalho tem a classificação, de acordo com \cite{gil2010metodos}, como desenvolvimento tecnológico,
onde, a partir de algumas necessidades e características, será desenvolvido um produto
de tecnologia para a sociedade.

\subsection{Referencial Teórico}
\label{sub:referencial_te_rico}
O referencial teórico foi elaborado utilizando pesquisas em livros digitais, artigos e
revistas indexados em três bases bibliográficas: \textit{Scopus}, \textit{IEEE}, \textit{Science Direct}. Todos estes
foram pesquisados através dos acessos disponibilizados pela Universidade de Brasília.
Alguns outros artigos e revistas foram escolhidos utilizando a técnica de
\textit{Snowball} para trás em no máximo, dois níveis.


\section{Estrutura de Monografia}
Este trabalho foi dividido em quatro capítulos principais. O capítulo 1 apresenta a introdução
do trabalho. O capítulo 2 trata sobre o que se tem de consolidado na academia em volta
dos temas que serão tratados. No capítulo 3 será apresentada a proposta do trabalho, que conterá
os desafios que este propõe tratar. O capítulo 4 tem o viés de contextualizar como será
desenvolvida e definida a proposta do trabalho.
