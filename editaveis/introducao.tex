\chapter[Introdução]{Introdução}
As redes socias tem se tornado extremamente populares na última década.
Várias
Redes Sociais despontaram e tomaram proporções grandes, tendo uma gama grande
de usuários participando destas. Alguns exemplos são: MySpace, Facebook, Twitter,
ByWorld, entre outras, que conseguiram milhões de usuários, onde a maioria integra
as suas funcionalidades com hábitos diários praticados pelos usuários.
Estas redes sociais estão repletas de tecnologias e funcionalidades diferentes,
trazendo características e suportando uma gama grande de interesses e práticas
entre as pessoas. Na maioria das vezes, apoiado na perssoazão e na presença social
dos indivíduos.

A Rede Social About (RSA) tem o propósito de dar transparência às personalidades de seus usuários, permitindo com que todos
estes saibam sobre qualquer aspecto sobre qualquer outro usuário, desde que ambos tenham aceitado os
termos de consentimento pré estabelecidos.

Já a Gamificação, para a definição de (Chou),  é o ato de cuidadosamente aplicar ao mundo
real e as atividades produtivas os elementos divertidos e envolventes dos jogos.
É a ação de enganjar e motivar os usuários a executarem alguma determinada
tarefa. 

Neste trabalho, será elaborado, definido e aplicado um framework de Gamificação na RSA 
para enganjar e motivar os seus usuários, utilizando uma abordagem propostar por (CHOU).


\section{Objetivos}
Foram separados os objetivos do trabalho entre gerais e específicos. Estes, serão
descritos a seguir.
\subsection{Objetivos Gerais}
Aplicar um framework de  gamificação adapitado para a Rede Social About.
\subsection{Objetivos Específicos}
\begin{itemize}
    \item Definição do Framework de Gamificação;
    \item Implementação da Gamificação na RSA;
    \item Coleta dos resultados da aplicação da Gamificação.
\end{itemize}
\section{Problema}
\section{Motivação}
\section{Metodologia}
\section{Estrutura de Monografia}
